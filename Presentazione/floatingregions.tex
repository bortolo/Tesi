
\begin{frame}
\frametitle{Il problema delle floating regions}
Con il termine \textbf{floating region} indichiamo quelle zone del semiconduttore caratterizzate da un preciso tipo di drogaggio, le quali non risultano direttamente contattate (ad esempio la zona N nel dispositivo PNP).

\vspace{0.5cm}

La risoluzione delle equazioni di DD nei dispositivi che presentano tali regioni ha prodotto una serie di \textbf{problematiche} che sono ancora oggeto di analisi.

\vspace{0.5cm}

\begin{itemize}
\item Esiste una procedura standard per affrontare questi dispositivi?
\item \`E necessaria una trattazione particolare delle condizioni di bordo della floating region?
\end{itemize}

\vspace{0.5cm}

Di seguito presentiamo i risultati pi\`u significativi.

\end{frame}

\begin{frame}
\frametitle{PNP senza contatto di BASE}
\begin{center}
\begin{figure}
\subfigure[Potential (V)]
{\includegraphics[scale=0.3]{Floating/LegendPNP_nogate18}}
\vspace{-0.5cm}
\end{figure}
\begin{figure}
\subfigure[Sdevice]
{\includegraphics[scale=0.33]{Floating/PNP_nogate_sdevice_18}}
\psp{40}
\subfigure[FEMOS]
{\includegraphics[scale=0.33]{Floating/PNP_nogate_tool_18}}
\end{figure}
\end{center}
\end{frame}

\begin{frame}
\frametitle{PNP senza contatto di base}
\begin{center}
\begin{columns}
\begin{column}{0.5 \paperwidth}
\begin{figure}
\subfigure[Portatori]
{\includegraphics[height=7cm,width=6cm]{Floating/PNP1e18_portatori}}
\end{figure}
\end{column}
\begin{column}{0.4 \paperwidth}
\begin{center}
\vspace{-2cm}

\begin{tikzpicture}
\useasboundingbox (-2.5,0) rectangle (1.5,2);
\draw [pattern=north west lines,pattern color = blue!40](-2,0.0) rectangle (-1,1);
\draw [pattern=north west lines,pattern color = red!40] (-1,0.0) rectangle (0,1);
\draw [pattern=north west lines,pattern color = blue!40] (0,0.0) rectangle (1,1);
\draw [dashed,ultra thick](-2.5,0.5)--(1.5,0.5);

\draw [fill=black] (-2.1,0.0) rectangle (-2,1);
\draw [fill=black] (1,0.0) rectangle (1.1,1);

\end{tikzpicture}


\footnotesize

\begin{alertblock}{ \footnotesize ATTENZIONE:}
Nella regione floating non sussiste la conservazione di carica!!
\end{alertblock}

\begin{itemize}
\item Il codice non va a convergenza, in particolare si avvicina alla soluzione per poi divergere.
\item  Abbiamo notato che nella zona di floating  \textit{non entrano} abbastanza portatori.
\end{itemize}

\end{center}
\end{column}
\end{columns}
\end{center}
\end{frame}

\begin{frame}
\frametitle{Conclusioni}
Abbiamo indagato su pi\`u fronti l'origine del problema, ma ancora non \`e stata individuata la causa:
\begin{itemize}
\item Non escludiamo la presenza di un possibile bug nel codice, tuttavia tale ipotesi \`e stata affrontata pi\`u volte negli utlimi giorni senza portare ad alcun esito.
\item Next step: abbiamo deciso di confrontare i risultati con il codice \texttt{COMPEL1D} usato a lezione, speriamo che questo confronto possa illuminarci.
\end{itemize}

Ancora non \`e chiara la natura di tale problema: \textbf{come fare ad isolarlo?} 

\end{frame}
\input{Floating}