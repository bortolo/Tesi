\chapter{Resolution of the system}

\section{Iteration algorithms}

The high nonlinear nature of the problem 	makes an analytical treatment very difficult, if not even impossible. For this reason, numerical schemes must be used to compute an approximate solution of the system \referenzaeq{eq: stationary problem}. Indisputably, the most used algorithms are \textit{the fully coupled Newton's method} and \textit{the decoupled Gummel map}. We spent only few words on the former as we implemented the latter, but first of all let us introduce a suitable linearization of the stationary DD equations. We can consider \referenzaeq{eq: stationary problem} in a more compact form:

\begin{equation}
\label{eq: abstract problem fully}
\vect{F(U)}=\vect{0}
\end{equation}

where:

\begin{equation}
\vect{U}:=[\varphi,n,p]^T \psp{30} \vect{F(U)}:=\left[ \begin{array}{c}
F_1(\vect{U}) \\
F_2(\vect{U}) \\
F_3(\vect{U})
\end{array}
\right]
\end{equation}

and having set:

\begin{align*}
F_1(\vect{U}) & = \nabla \cdot (-\epsilon \nabla \varphi) - q(p-n+N_D^+-N_A^-) \\
F_2(\vect{U}) & = \nabla \cdot ( - q\mu_n n \nabla \varphi + qD_n \nabla n )-qR \\
F_3(\vect{U}) & = \nabla \cdot (- q\mu_p p \nabla \varphi - qD_p \nabla p )+qR
\end{align*}

\vspace{0.1cm}

\subsection{Fully Coupled Newton's Method}

Basically the fully coupled Newton's method is an extension to differential operators of the well known Newton's method for the search of a zero for real functions ($f:\mathbb{R}\rightarrow \mathbb{R}$). In fact the vector function $\vect{F}$ is a nonlinear differential operator and the associated problem which we intend to resolve is: given a functional space $V$ and the operator $\vect{F}:V\rightarrow V$, find $\vect{U}\in V$ such that \referenzaeq{eq: abstract problem fully} is satisfied.
Just for the moment we don't care about the identity of $V$, which we will treat in details in the next section; now we are able to define  the abstract Newton's Method for the iterative solution of problem \referenzaeq{eq: stationary problem}:

\mybox{
\vspace{0.1cm}
Given $\vect{U}^0\in V$, for all $k\geq 0$ until convergence, solve the following linearized problem: 

\begin{equation}
\label{eq: abstract newton's method}
\begin{array}{c}
\vect{F'}(\vect{U}^k)\delta\vect{U}^k=-\vect{F}(\vect{U}^k) \\
\vect{U}^{k+1} = \vect{U}^k + \delta \vect{U}^k
\end{array}
\end{equation}
where $\vect{F'}$ is the Jacobian matrix of $\vect{F}$, whose $(i,j)$-th entry represents the Frech\`et derivative of the $i$-th row with respect to the $j$-th variable.
}{\textbf{Fully Coupled Newton's Method}}

\vspace{0.1cm}

The main advantage of this approach is without doubt the existence of the sequent theorem:

\begin{Teorema}
\label{theorem: newton convergence}
Let  $\vect{U}\in V$ be a solution of problem \referenzaeq{eq: abstract problem fully}. Assume that $\vect{F'}$ is Lipschitz continuos in the ball $\mathcal{B}(\vect{U},\delta)$, i.e., that there exists  $K>0$ such that:
\begin{equation}
||\vect{F'}(\vect{v})-\vect{F'}(\vect{z})||_{L(V,V)} \leq K ||\vect{v}-\vect{z}||_V \psp{10} \forall \vect{v},\vect{z} \in \mathcal{B}(\vect{U},\delta), \, \vect{v}\neq \vect{z}
\end{equation}
Then there exists in correspondence $\delta '>0$, with $\delta '\leq\delta$, such that for all $\vect{U}^0 \in \mathcal{B}(\vect{U},\delta ')$ the sequence $\left\{ \vect{U}^k \right\}$ generated by \referenzaeq{eq: abstract newton's method} converges quadratically to $\vect{U}$, i.e., there exists $C>0$ such that, for a suitable $k_0\geq 0$ we have:
\begin{equation}
\label{eq: convergece newton}
||\vect{U}-\vect{U}^{k+1}||_V\leq C||\vect{U}-\vect{U}^k||_V^2 \psp{15} \forall k\geq k_0
\end{equation}
\end{Teorema}

Even the presence of this impressive result there are are several issues which must be evaluated before move toward this way of resolution:
\begin{itemize}
\item the jacobian matrix $\vect{F'}$ is often quite ill-conditioned and needs appropriate scaling and balancing in order to avoid problems associated with round-off error;
\item to ensure convergence of the Newton iterative process \referenzaeq{eq: abstract newton's method}, it is particularly important to ensure a very good initial guess for the unknown variables $\vect{U}$;
\item if a direct solver is adopted (for example based on the LU factorization method), the number of loating point operations is of the order of  $N_{dofs}^3$, where $N_{dofs}$ is the number of degree of freedom used for the numerical approximation.
\end{itemize}

The fully coupled method is not the only possible approach, and many of the above points will be fixed adopting different methods, however theorem \referenzaeq{theorem: newton convergence} guarantees the best error convergence estimation.

\subsection{Gummel map algorithm}

In 1964 H. K. Gummel proposed an orginal iterative algorithm in order to solve the system \referenzaeq{eq: stationary problem} in a semiconductor device in one spatial dimension. Today the Gummel decoupled iterative algorithm has become a milestone in contemporary device simulation in industrial software for the design and analysis of semiconductor devices.

The main idea of the algorithm is to move the nonlinearity to the Poisson equation only and once obtained the electric potential profile, both continuity equations are linearized.  More precisely give an initial guess for $\varphi$, $n$ and $p$, the functional iteration consists in the succesive solution of the nonlinear Poisson's equation (NLP) in a inner loop (Newton's method is applyed for this equation) and of the two linearized continuity equations (DD electrons and DD holes). 
A concisely scheme is presented in \figref{fig: gummel map}.

Unfortunately there isn't any convergence result for this method like \referenzaeq{theorem: newton convergence}, although there are several advantages which make Gummel map algorithm preferable to the Fully Coupled Newton's Method.
In fact simulations experience shows that the Gummel process is much more insesitive to the choice of the initial guess than Newton's method. This is particularly important in multidimensional problems where it is far from trivial to design a good starting point for initializing in a favorable manner.

Another attractive feature is the reduced computational and memory stage cost: at each iteration step, the Newton algorithm requires assembling a Jacobian matrix of size $3N_{dof}\times 3N_{dof}$, while the Gummel algorithm requires the successive solution of three problems, each one of size equal to $N_{dof}\times N_{dof}$.

\subsubsection{FEMOS Gummel map}
After this short presentation of the qualities and the drawbacks of the Gummel decoupled algorithm, we propose our functional iteration. For the sake of simplicity let us consider some useful hypotesis:






\begin{figure}[!h]
\begin{center}
\begin{tikzpicture}
[scale=1.2]

\tikzstyle{NLP}=[rectangle,text width=3cm, align=center,draw, fill=gray!40];
\tikzstyle{DD}=[rectangle,text width=3cm,align=center,draw,fill=gray!10];
\tikzstyle{Normal}=[rectangle,fill=white];
\tikzstyle{CYC}=[diamond,draw];
\useasboundingbox (0,1) rectangle (8,9.5);
\draw [thick] (-0.5,1.0) rectangle (8.5,9.5);
%main boxes
\node [Normal] (v0) at (1,8) {$[\varphi,n,p]_{start}$};
\node [NLP] (v1) at (4,8) {\large NLP};
\node [CYC] (v2) at (4,7) {\small k};
\node [DD] (v3) at (4,5.5) {\large DD electrons};
\node [DD] (v4) at (4,4.0) {\large DD holes};
\node [CYC] (v5) at (4,2.5) {\small i};

%collegamenti
\draw [thick,->] (v0) -- (v1);
\draw [thick,->] (v1) -- (v2);
\draw [thick,->] (v2) -- (v3);
\draw [thick,->] (v3) -- (v4);
\draw [thick,->] (v4) -- (v5);
\draw [thick,->] (v5)--(7,2.5)--(7,9)--(4,9)--(v1);
\draw [thick,->] (v2)--(6,7)--(6,8)--(v1);
\draw [thick,->] (v5)--(4,1.5);

%note
\node [Normal] (v6) at (5.5,7.3) {$\varphi^{k+1}$};
\node [Normal] (v6) at (4.5,6.2) {$\varphi_{i+1}$};
\node [Normal] (v6) at (4.5,4.7) {$n_{i+1}$};
\node [Normal] (v6) at (4.5,3.2) {$p_{i+1}$};
\node [Normal] (v0) at (5.0,1.8) {$[\varphi,n,p]_{end}$};
\end{tikzpicture}
\caption{Gummel map algorithm}
\label{fig: gummel map}
\end{center}
\end{figure}


\begin{itemize}
\item we consider a polygonal simulation domain  in $\mathbb{R}^3$, denoted with $\Omega$; we indicate also with $\Omega_s$ the subset of $\Omega$ characterized by semiconductor material;
\item the domain is formed only by semiconductor and/or oxide material (typically we consider silicon and $SiO_2$);
\item contacts are assumed to be ideal, i.e. they are equipotential surfaces and no voltage drop occurs at the interface between the contact and the neighbouring material.
\end{itemize}
Take into account these considerations the appropriate Gummel map in our cases is the sequent:




\mybox{
Given $n^0$ and $p^0$, $\forall i$ until convergence:

\vspace{0.5cm}

(Step 0) Compute $\varphi_n^i$ and $\varphi_p^i$ with \referenzaeq{eq: n density mb} and \referenzaeq{eq: p density mb} and give a suitable initial guess for $\varphi^0_i$.

\vspace{0.5cm}

(Step 1) Solve the nonlinear Poisson equation over all the domain (NLP):

{
\footnotesize
\begin{equation}
\label{eq: NLP system}
\begin{cases}
\nabla \cdot (-\epsilon \nabla \varphi) + n_i\left( exp\left(\dfrac{\varphi-\varphi_n^i}{V_{th}}\right) - exp\left(\dfrac{\varphi_p^i-\varphi}{V_{th}}\right) \right)  =  q(N_D^+-N_A^-) & in \psp{3} \Omega_s \\
\nabla \cdot (-\epsilon \nabla \varphi)  =  0 & in \psp{3} \Omega / \Omega_s
\\
\varphi = \varphi_D & on \psp{3} \Gamma_D
\\
\nabla \varphi \cdot \vect{n} = 0 & on \psp{3} \Gamma_N 
\end{cases}
\end{equation}
}

Set $\varphi^i=\varphi$.

\vspace{0.5cm}

(Step 2) Solve the Linear Electron Contintuity Equation (LEC):
{
\small
\begin{equation}
\label{eq: LEC system}
\begin{cases}
 \nabla \cdot ( - q\mu_n n \nabla \varphi^i + qD_n \nabla n ) = qR(n^{i-1},p^{i-1}) & in \psp{3} \Omega_s
 \\
 n = n_D & on \psp{3} \Gamma_D
 \\
 \nabla n \cdot \vect{n} = 0 & on \psp{3} \Gamma_N
\end{cases}
\end{equation}
}
Set $n^i=n$.

\vspace{0.5cm}

(Step 3) Solve the Linear Hole Contintuity Equation (LHC):
{
\small
\begin{equation}
\label{eq: LHC system}
\begin{cases}
\nabla \cdot (- q\mu_p p \nabla \varphi^i - qD_p \nabla p ) =  -qR(n^{i-1},p^{i-1}) & in \psp{3} \Omega_s
\\
 p = p_D & on \psp{3} \Gamma_D
 \\
 \nabla p \cdot \vect{n} = 0 & on \psp{3} \Gamma_N
\end{cases}
\end{equation}
}
Set $p^i=p$.

\vspace{0.5cm}

(Step 4) If the convergence criterion is satisfied break, otherwise restart from step 0.

}{\textbf{FEMOS Gummel Map}}

\textcolor{red}{commenti a queste ipotesi magari si possiamo lavorarci sopra}
The second hypotesis excludes from our simulations only metal materials but there isn't any problem if some subset of the domain is formed by metal, it's possbile implement a more general formulation of the Poisson equation which it's presented in \textcolor{red}{referenza silvia}.
We fixed the third hypostesis because Robin conditions are not performed yet for this part of the code, but a suitable extension it's a n exercise.
 
\textcolor{red}{Dobbiamo parlare delle slotboom variables e della possibilità di usare i quasi fermi?????}
It's interesting note that carrier densities are not the only useful variables. As we introduced in section \referenzaeq{subsub:driftdiffusion transport}, current density may be represented in many different ways.  We underlie this because for example one can solve system \referenzaeq{eq: stationary problem} with Slotboom variables, without any change in the structure of the Gummel map.


\chapter{Finite element discretization}

In this section we describe the finite element discretization of the differential subproblems involved in the Gummel map previously introduced. Moreover for each kind of PDE problem we give a briefly presentation of the well-posedness analysis. 


\section{Weak Formulation}

In this section we'll introduce the weak formulation used for the above equations accordingly to the classical displacement approach. The generic weak formulation is find $u\in V$:
\begin{equation}
\label{eq: generic weak formulation}
a(u,v) = F(v) \psp{10} \forall v \in V
\end{equation}

We work with the sequent family of Sobolev functional spaces:

\begin{align}
H^m(\Omega) & := \left\{  v \in L^m(\Omega) : D^\alpha \in L^m(\Omega) \, \forall \alpha, |\alpha|\leq m\right\}
\\
H^m_{\Gamma}(\Omega) & := \left\{  v \in H^m(\Omega) : v|_{\Gamma} = 0 \right\}
\end{align} 

provided with the usual norm and seminorm $|| v ||_{m,\Omega}$ and $|v|_{m,\Omega}$.
In order to prove the existence and uniqueness of the solutions of the variational problems which will be introduced in the next sections, we apply the Lax-Milgram theorem \cite{salsa:EDP} to the weak formulations.

\subsubsection{Nonlinear Poisson Equation}

Using the definition of the Frech\`et derivative one can calculated the Jacobian matrix of problem \referenzaeq{eq: NLP system}, which in this case has $1\times 1$ dimensions. In contrast with the Fully Coupled Newton Method \referenzaeq{eq: abstract newton's method}, the functional iteration has a unique variable which is $\varphi$:

\begin{equation}
[\varphi] \rightarrow F([\varphi])
\end{equation}

Therfore given $\varphi^0$, the Newton step for the linearized non linear Poisson equation looks as follows (remember that carrier densities are computed with the Maxwell-Boltzmann approximation):

\begin{equation}
\label{eq: newton step NLP}
\begin{cases}

\nabla \cdot (-\epsilon_s \nabla \delta\varphi^k) 
+   \dfrac{1}{V_{th}} \sigma_s^k\delta\varphi^k 
 =  f_s^k & in \psp{3} \Omega_s
  \\
\nabla \cdot (-\epsilon_{ox} \nabla \delta\varphi^k) =  f_{ox}^k & in \psp{3} \Omega / \Omega_s 
\\
\delta \varphi^k = 0 & on \psp{3} \Gamma_D 
\\
\nabla \delta \varphi^k \cdot \vect{n} = 0 & on \psp{3} \Gamma_N
\\
\varphi^{k+1}=\varphi^k+\delta \varphi^k
\end{cases} 
\end{equation}

having set,

\begin{align*}
\sigma_s^k(\varphi^{k},\varphi_n^{i},\varphi_p^{i}) & =q(p(\varphi^k,\varphi_p^i)-n(\varphi^k,\varphi_n^i))
\\
f_s^k(\varphi^k,\varphi_n^i,\varphi_p^i) & = \nabla \cdot (-\epsilon \nabla \varphi^k) + q\left[ p(\varphi^k,\varphi_p^i)-n(\varphi^k,\varphi_n^i)) + N_D^+-N_A^- \right]
\\
f_{ox}(\varphi^k) & = \nabla \cdot (-\epsilon \nabla \varphi^k) 
\end{align*}

We remark the importance to give an appropriate $\varphi^0$ which respects Dirichlet boundary condition $\varphi_D$, although the problem can't satisfied in strong manner such condition. Moreover with suitable intial guess we intend also a function which hopefully is near as possible at the attractive region of the problem \referenza{theorem: newton convergence}.
The first two equations of system \referenzaeq{eq: newton step NLP} constitute a classical DR (Diffusion-Reaction) problem in $\Omega$, respect the variable $\delta \varphi^i$. We are able to consider a unique piecewise electric permettivity, force term and reaction defined as follows:
\begin{align*}
\epsilon & = \epsilon_s \mathcal{I}_{\Omega_s} + \epsilon_{ox} \mathcal{I}_{\Omega / \Omega_s} \\
f & = f_s \mathcal{I}_{\Omega_s} + f_{ox} \mathcal{I}_{\Omega / \Omega_s} \\
\sigma & = \sigma_s \mathcal{I}_{\Omega_s}
\end{align*}

The more generalized form of \referenzaeq{eq: newton step NLP} reads as follows:
\begin{equation}
\label{sys: NLP general problem linearized}
\left\{
\begin{array}{rcll}
\nabla \cdot (-\epsilon \nabla \delta \varphi^k) + \sigma^k \delta \varphi^k & = &  f^k & \psp{15} in \psp{2} \Omega \\
\delta \varphi^k & = & 0 & \psp{15} on \psp{2} \Gamma_D \\
\nabla \delta \varphi ^k\cdot \vect{n} & = & 0 & \psp{15} on \psp{2} \Gamma_N
\\
\varphi^{k+1} & = & \varphi^k + \delta \varphi^k
\end{array}
\right.
\end{equation}

The well-posedness of such problem is ensured by several (and physical) hypotesis:
\begin{itemize}
\item $\epsilon \in L^{\infty}(\Omega)$ and $\exists m$ s.t. $0 < m \leq \epsilon$ (a.e.) in $\Omega$;
\item  $\sigma \in L^{\infty}(\Omega)$ and $\exists m$ s.t. $0 < m \leq \sigma$ (a.e.) in $\Omega_s$.
\end{itemize}

The relative weak formulation reads as follows: given $f \in L^2(\Omega)$ find $\delta\varphi \in H^1_{\Gamma_D}(\Omega)$ such that 

\begin{equation}
\label{eq: NLP weakformulation}
\int_{\Omega} \epsilon \nabla \delta\varphi \nabla v \, d\Omega + \int_{\Omega} \sigma^{(k)}\delta \varphi v \, d\Omega = \int_{\Omega} f^{(k)}v \, d\Omega \psp{15} \forall v \in H^1_{\Gamma_D}(\Omega)
\end{equation}

For the well-posedness of \referenzaeq{eq: NLP weakformulation} is useful define the sequent quntities:
\begin{equation*}
\begin{array}{ll}
\epsilon_M = max_{\Omega} \epsilon & \epsilon_m = min_{\Omega} \epsilon \\
\sigma_M = max_{\Omega} \sigma & \sigma_m = max_{\Omega} \sigma = 0 \\
\end{array}
\end{equation*}
Take into account the above hypotesis one can demonstrate:
\begin{itemize}
\item \textbf{Continuity},
\begin{equation*}
\begin{array}{ll}
\forall u,v \in H^1_{\Gamma_D} &\\ \\
|\int_{\Omega} \epsilon \nabla u \nabla v + \int_{\Omega} \sigma^{(k)}u v| 
& \leq \epsilon_{M} ||\nabla u ||_{L^2} || \nabla v ||_{L^2} +  \sigma_{M} ||u ||_{L^2} ||v ||_{L^2} 
\\
& \leq max\{\epsilon_{M}, \sigma_{M} \}  
\left( ||\nabla u ||_{L^2} || \nabla v ||_{L^2} +   ||u ||_{L^2} ||v ||_{L^2} \right)
\\
& \leq max\{\epsilon_{M}, \sigma_{M} \}  
||u ||_{H^1_{\Gamma_D}} || v ||_{H^1_{\Gamma_D}}
\end{array}
\end{equation*}

\item \textbf{Coercivity,}
\begin{equation*}
\begin{array}{ll}
\forall u \in H^1_{\Gamma_D} &\\ \\
|\int_{\Omega} \epsilon \nabla u \nabla u + \int_{\Omega} \sigma^{(k)}u^2| 
& \geq \epsilon_{m} ||\nabla u ||_{L^2}^2  +  \sigma_{m} ||u ||_{L^2}^2 
\\
& =  \epsilon_{m} ||\nabla u ||_{L^2}^2 
\\
& = \epsilon_{m} |\nabla u |_{H^1_{\Gamma_D}}^2 
\end{array}
\end{equation*}

\item \textbf{Continuity of the functional,}
\begin{equation*}
\begin{array}{ll}
|\int_{\Omega} f^{(k)} v |
& \leq ||f^{(k)} ||_{L^2}||v ||_{L^2} \psp{15} \forall v \in H^1_{\Gamma_D}
\end{array}
\end{equation*}
\end{itemize}

Then we can state that there exists a unique solution of the lineariezed Poisson equation.

\subsubsection{Continuity Equation}

Without loss of generality we can consider only the electron continuity equation. Problem \referenzaeq{eq: LEC system} is a classical diffusion-advection-reaction (DAR) problem written in conservative form. We will treat this PDE's equation likewise Poisson equation with the standard displacement weak formulation.

As we descirbed in section \referenza{subsection: RG} recombination/generation phenomenon could produce an additional reaction term so in order to include these cases into our weak formulation we denote the general recombination/generation term as follows:

\begin{equation}
R_n = \sigma n - f
\end{equation}

Furthermore the demonstration of the well-posedness became easier if we rewrite system \referenzaeq{eq: LEC system} with Slotboom variables:
\begin{equation}
\small
\left\{
\begin{array}{rcll}
 \nabla \cdot \left( - q D_n exp(\varphi/V_{th}) u_n \right) + \sigma exp(\varphi/V_{th}) u_n& = & f  & \psp{15} in \psp{2} \Omega \\
u_n & = &  n_D exp(-\varphi/V_{th}) & \psp{15} on \psp{2} \Gamma_D \\
\nabla u_n \cdot \vect{n} & = & 0 & \psp{15} on \psp{2} \Gamma_N
\end{array}
\right.
\end{equation}

In order to simplify the notation we summerize the coefficients of the above system:
\begin{align*}
\bar{D_n} & = q D_n exp(\varphi/V_{th}) \\
\bar{\sigma}  & = \sigma exp(\varphi/V_{th})
\end{align*}
The existence and uniqueness of the unkown variable $u_n$ enures the same properties on $n$, thanks to the univocal relation between $u_n$ and $n$.
What we obtained with this smart sostitution is a system totally similar to the linearized Poisson problem and consequently we should make similar hypotesis onthe coefficients:
\begin{itemize}
\item $\bar{D_n} \in L^{\infty}(\Omega)$ and $\exists m$ s.t. $0 < m \leq \bar{D_n}$ (a.e.) in $\Omega$;
\item  $\bar{\sigma} \in L^{\infty}(\Omega)$ and $\exists m$ s.t. $0 < m \leq \bar{\sigma}$ (a.e.) in $\Omega_s$.
\end{itemize}
Finally we can write the weak formulation: given $u_{n_D} \in H^{1/2}(\Gamma_D)$ and $f \in L^2(\Omega)$ find $u_n \in H^1(\Omega)$ such that:

\begin{equation}
\label{eq: LEC weakformulation}
\int_{\Omega} \bar{D_n} \nabla u_n \nabla v \, d\Omega + \int_{\Omega} \bar{\sigma} u_n v \, d\Omega = \int_{\Omega} f v \, d\Omega \psp{15} \forall v \in H^1_{\Gamma_D}(\Omega)
\end{equation}


\section{Geometrical discretization}

In view of the Galerkin finite element discretization of probems \referenzaeq{eq: NLP system}, \referenzaeq{eq: LEC system} and \referenzaeq{eq: LHC system}, we introduce some useful notations.

We let $\bar{\Omega} =  \bigcup \bar{K}$ be a partition $\mathcal{T}_h$ of the domain $\Omega$ into tetrahedral elements $K$ of volume $|K|$, i.e. we suppose that there exists a constant $\delta>0$ such that:
\begin{equation}
\label{eq: mesh regular condition}
\dfrac{h_K}{\rho_K} \leq \delta \psp{15} \forall K \in \mathcal{T}_h
\end{equation}

where $h_k=diam(K)=max_{x,y\in K}|x-y|$ and $\rho_K$ is the diameter of the sphere inscribed in the tetrahedral $K$. Condition \referenzaeq{eq: mesh regular condition} is the so called \textit{mesh regularity condition} \cite{quarteroni:modnum} and it ensures an istropic partition.
We denote with $\mathcal{E}_h$, $\mathcal{V}_h$ and $\mathcal{F}_h$ the set of all the edges, verteces and faces  
of $\mathcal{T}_h$ respectively, and for each $K\in \mathcal{T}_h$ we denote by $\partial K$ and $\vect{n}_{\partial K}$ the boundary of the element and its outward unit normal.
  
\section{Numerical approximation}

Let us explain briefly the Galerkin method for a general weak formulation like \referenzaeq{eq: generic weak formulation}. The general form of the approximate Galerkin problem reads as find $u_h\in V_h$:
\begin{equation}
\label{eq: general galerkin problem}
a(u_h,v_h) = F(v_h) \psp{10} \forall v \in V_h
\end{equation}


where $V_h$ is a suitable finite space such that $V_h \subset V$ and approximation preperties are satisfied \textcolor{red}{referenza}.
For our purposes let us introduce the general finite element spaces of the polynomial element-wise defined functions used to approximate $H^1(\Omega)$ and $H^1_{\Gamma_d}(\Omega)$:

\begin{align}
X^r_h &:= \{v_h \in C^0(\bar{\Omega}): v_h|_K\in \mathbb{P}_r,\forall K \in \mathcal{T}_h \} \\
X^r_{h,\Gamma_D} & := \{ v_h \in X^r_h: v_h|_{\Gamma_D} = 0 \} 
\end{align}

More precisely during implementation we decide to set $V_h = X^1_{h,\Gamma_D}$. 
We denote by $\{ \psi_j \}_{j=1}^{N_h} $ the Lagrangian basis of the space $X^1_{h,\Gamma_D}$ and $N_h$ being the finite dimension of the space. Naturally  as $u_h \in X^1_{h,\Gamma_D}$ we have:
\begin{equation}
u_h = \sum_{j=1}^{N_h} u_j \psi_j
\end{equation}

Furthermore now it is sufficient verified Gakerkin problem \referenzaeq{eq: general galerkin problem} only for each basis functions $\psi_j$. The result of the complete discretization is find $u_j$, with $j = 1, \, . \, . \, . \, , \, N_h$ such that:

\begin{equation}
\sum_{j=1}^{N_h}u_ja(\psi_j,\psi_i) = F(\psi_i) \psp{10} \forall i = 1, \, . \, . \, . \, , \, N_h
\end{equation}

In order to implement this routine it's useful make explicit the subdivsion of the bilinear form on the element of the partition $\mathcal{T}_h$:

\begin{equation}
\sum_{j=1}^{N_h}u_j\sum_{K\in \mathcal{T}_h}a_K(\psi_j,\psi_i) = \sum_{K\in \mathcal{T}_h}F_K(\psi_i) \psp{10} \forall i = 1, \, . \, . \, . \, , \, N_h
\end{equation}

\subsection{Nonlinear Poisson Equation}


As regards the linearized Poisson equation we have:
\begin{equation}
a(\psi_j,\psi_i)  = \int_{\Omega} \epsilon \nabla \psi_j \nabla \psi_i \, d\Omega + \int_{\Omega} \sigma^{(k)}\psi_j \psi_i \, d\Omega 
\end{equation}
and the relative restriction on the element is
\begin{equation}
\label{eq: bilinear local discrete}
a_K(\psi_j,\psi_i)  = \int_{K} \epsilon \nabla \psi_j \nabla \psi_i \, dK + \int_{K} \sigma^{(k)}\psi_j \psi_i \, dK
\end{equation}

Equation \referenzaeq{eq: bilinear local discrete} it's formed by two distinct contributions, the former indentifies the diffusive contribution and generates the so called \textit{stifness matrix}, while the latter refers to the reaction and generates the \textit{mass matrix}.


It's possible adopt some simplifications which makes easier the treatment of these integrals. First of all consider that the diffusive coefficient $\epsilon$ can be a rapidly varying function and one could choose to integrate it with a consequently grow up of the computational cost, or it's possible use a suitable average on each elment of this quantity. In view of further discussions of this issue, for each set $S \subset \Omega$ having measure $|S|$, we introduce the following averages of a given function $g$ that is integrable on $S$:

\begin{equation*}
\mathcal{M}_S(g) = \dfrac{\int_S g \, dS}{|S|} \psp{15} \mathcal{H}_S = (\mathcal{M}_S(g^{-1}))^{-1} 
\end{equation*}

Notice that $\mathcal{M}_S$ is the usual average, while $\mathcal{H}_S$ is the \textit{harmonic average}. It is well-known that the use of tha harmonic average provides a superior approximation performance than the usual average \textcolor{red}{referenza}.

Takling about the reaction contribution, thus we decide to use $X^1_{h,\Gamma_D}$,  we can't expect a better priori estimation error on the solution, than the first order respect the discretization step $h$ \textcolor{red}{referenza}. This implies which is not necessary and useful use an high order quadrature, for example the trapezoidal rule is enough accurate in this case. 
The main conseguence of using trapezoidal quadrature rule is that extra-diagonal elements of the mass-matrix disappear.
This technique is well known as \textit{lumping procedure} applied on the mass-matrix \textcolor{red}{referenza}.
Finally we obtain the contributions of the local system matrix $A_K^{(k)}$:

\begin{equation}
[A_K]_{ij}^{(k)}  = \mathcal{H}_K(\epsilon)\left[
L_{ij}
\right]
+
\dfrac{|K|}{4}diag(\sigma^{(k)}_i)
\end{equation}

having set

\begin{equation}
L_{ij} = \int_K \nabla \psi_i  \nabla \psi_j \, d\Omega
\end{equation}

Here follows the construction of the right hand side, based on the local contribution approximated with trapezoidal rule:

\begin{equation}
[F_K]_i^{(k)} = \int_K f^{(k)} \psi_i \, dK \simeq f^{(k)}_i |K| / 4 
\end{equation}

We can assemble the local contributions for each element $K$ in the global matrix $A$: let $I$ be the global index of a generic vertex belonging to the partition $\mathcal{T}_h$, we denote by $\mathcal{J}_K: \mathcal{V}_{\mathcal{T}_h} \rightarrow \mathcal{V}_{K}$ the map which connects $I$ to its corresponding local index $i=1, \, . \, . \, . \, , 4$ in the element $K$. Then we have 

\begin{equation}
A_{IJ}^{(k)} = \sum_{\substack{\forall K \in \mathcal{T}_h s.t. \\ \mathcal{J}_K(I),\mathcal{J}_K(J) \subset \mathcal{V}_K}}
 [A_K]_{ij}^{(k)}
\end{equation}

analogously for the force term $\vect{b}^{(k)}$:

\begin{equation}
b_{I}^{(k)} = \sum_{\substack{\forall K \in \mathcal{T}_h s.t. \\ \mathcal{J}_K(I) \subset \mathcal{V}_K}}
 [F_K]_{i}^{(k)}
\end{equation}

Once we have built the global matrix $A^{(k)}$ and the global vector $\vect{b}^{(k)}$ we need to take into account the essential boundary conditions. In fact the displacement formulation is a primal formulation which forces Dirichlet boundary condition in a strong way. Therefore we have to modify the algebraic system and to do that there are several ways. We choose the \textit{diagonalization} techinique which does not alter the matrix pattern nor introduce ill-conditioning for the system.  Let $i_D$ be the generic index of a Dirichlet node, we denote by $[\delta \varphi_{D}]_i$ (which in this case is equal to zero) the known value of the solution $\delta \varphi $ at the node. We consider the Dirichlet condition as an equation of the form $a [\delta \varphi]_i = a [\delta \varphi_{D}]_i$, where $a\neq 0$ is a suitable coefficient. In order to avoid degrading of the global matrix condition number, we take $a$ equal to the diagonal element of the matrix at the row  $i_D$.

Finally we have completed the discretization of (Step 1), which reads as follows:

\begin{equation}
\label{eq: NLP discretizated}
\left\{
\begin{array}{rcl}
A^{(k)}\vect{\delta \varphi}^k & = & \vect{b}^{(k)} \\
\vect{\varphi}^{k+1} & = & \vect{\varphi}^k +  \vect{\delta \varphi}^k 
\end{array}
\right.
\end{equation}

As every iteration procedure, problem \referenzaeq{eq: NLP discretizated} needs a suitable convergence break criterion. A good method is based on cheking the satisfaction of the fixed point equation \referenzaeq{eq: abstract problem fully} by the $k$-th solution, then the inner loop of the Gummel Map reads as: given a tollerance $toll>0$ solve problem \referenzaeq{eq: NLP discretizated} untill:
\begin{equation}
||\vect{b}(\varphi_k)||_2 > toll
\end{equation}

where with $||\cdot ||_2$ we indicate the usual Euclideian norm for a vector.
\subsubsection{Damping}

Even if we have important results of convergence about the Newton's method, the system \referenzaeq{eq: NLP discretizated} may be affected by diffulties on the convergence velocity.
The main problem associated with the classical Newton method is the tendency to overestimate the length of the actual correction step for the iterate. This phenomenon is frequently termed overshoot. In the case of the semiconductor equations this overshoot problem has often been treated by simply limiting the size of the correction vector ($\delta \varphi$) determined by Newton's method. The usual established modifications to avoid overshoot are given by the seguent formulations:


\begin{align}
\tilde{A}(\varphi_k)&=\dfrac{1}{t_k}A(\varphi_k) \label{eq: NLP mod used} \\
\tilde{A}(\varphi_k)&=s_kI+A(\varphi_k) \label{eq: NLP mod not used}
\end{align}

$t_k$ and $s_k$ are properly chosen positive parameters. During the implementation of the code we chose \referenzaeq{eq: NLP mod used} method. Note that for $t_k=1$, $s_k=0$ these modified Newton methods reduce to the classical Newton method. We have now to deal with the question how to choose $t_k$ or $s_k$ such that the modified Newton methods exhibit superior convergence properties compared to the classical Newton method.
For the case \referenzaeq{eq: NLP mod used} there's a simple criterion suggested by Deuflhard \cite{selberherr:SimSem} $t_k$ is taken from the interval $(0,1]$ in such a manner that for any norm,
\begin{equation}
\label{eq: extended criterion}
||A(\varphi_k)^{-1}\vect{b}(\varphi_k-t_kA(\varphi_k)^{-1}\vect{b}(\varphi_k))||<||A(\varphi_k)^{-1}\vect{b}(\varphi_k)||
\end{equation}

Condition \referenzaeq{eq: extended criterion} guarantees that the correction of the k-th iterate is an improved approximation to the final solution, in other words the residual norm can only descents.
This condition can be easily evaluated only if the Jacobian matrix is factored into triangular matrices because the evaluation of the argument of the norm on the left hand side of \referenzaeq{eq: extended criterion} is then reduced to a forward and backward substitution and the evaluation of $\vect{b}(\varphi)$. Although we use an iterative methods (BCG solver) which implies serious diffuclties to the application of the above criterion. Another valid possibility is to use the main diagonal of $A(\varphi_k)$, denoted as $D(\varphi_k)$:
\begin{equation}
\label{eq: easy criterion}
||D(\varphi_k)^{-1}\vect{b}(\varphi_k-t_kD(\varphi_k)^{-1}\vect{b}(\varphi_k))||<||D(\varphi_k)^{-1}\vect{b}(\varphi_k)||
\end{equation}

This is the criterion developed in our code. However the value to use for $t_k$ is a question of trial and error. Frequently one chooses the following sequences:

\begin{align}
t_k & = \dfrac{1}{2^i} \\
t_k & = \dfrac{1}{2^{\dfrac{i(i+1)}{2}}}  
\end{align}

obiuvsly $i$ is the subiterations of damping reached when satisfied \referenzaeq{eq: easy criterion}. Sufficiently close to the solution \referenzaeq{eq: extended criterion} (and so \referenzaeq{eq: easy criterion}) will be satisfied with $t_k=1$ so that the convergence properties of the classical Newton method are recovered.
 

\subsection{Continuity Equation}

Let us write the discretized electron continuity equation in a suitable generic form on the $K$-th element:
\begin{equation}
\label{eq: LEC discretized general}
\left\{
\begin{array}{rcl}
a_h^K(n_h,v_h) & = & \int_{K} J_h^K(n_h) \nabla v_h \, dK + \int_{K} \sigma n_h v_h \, dK 
\\
\\
F(v_h)^K & = & \int_K f v_h \, dK
\\
\\
J_h^K & = & D_K(\bar{D}_n) \nabla  (e^{(-\varphi / V_{th})}n_h)
\end{array}
\right.
\end{equation}

Notice that we came back on carrier density variable, because 
\textcolor{red}{finire spiegazione sulla difficolta numerica variabili di slotboom}

We want to extend now the discussion about the treatment of the diffusive coefficient. The matrix $D_K(\bar{D}_n) \in \mathbb{R}^3$ can be characerized in three different ways:

\begin{equation}
D_K(\bar{D}_n) = \left\{ 
\begin{array}{l}
\mathcal{M}_K(\bar{D}_n) \, \mathcal{I}
\\
\\
\mathcal{H}_K(\bar{D}_n) \, \mathcal{I}
\\
\\
 \dfrac{1}{|K|}\sum_{i=1}^6 \mathcal{H}_{e_i}(\bar{D}_n) |e_i| s_i \vect{t}_i \vect{t}_i^T
\end{array}
\right.
\end{equation} 

These different approaches in the computation of the average of the diffusion coefficient are responsible for the quite different numerical perfomance of the relative methods.
We already presented the standard average and the armonic average and we discussed briefly the advantages of them. 
Now we are interested in the latter method which  is an exponentially fitted treatment of $\bar{D}_n$ over the entire subdomain $K$. 
From \referenzaeq{eq: LEC discretized general} we immeditaly obtain:

\begin{equation}
\label{eq: exp fitted flux}
\vect{J}_h^K = \dfrac{1}{|K|}\sum_{i=1}^6 |e_i| s_i j_{e_i} \vect{t}_i 
\end{equation}

Having defined the flux vector over $K$ in the form \referenzaeq{eq: exp fitted flux}, it is clearly possible to construct a fimily of Galerkin finite element approximations for the continuity equations by a proper choice of the quantities $j_{e_i}$. 

\subsubsection{Edge flux density computation}

We introduce an approximate flux density computation along the edges of $\partial K$ based on the classical one dimensional \textit{Sharfetter-Gummel} method \textcolor{red}{referenza o anticipazione?}. For each edge $e_i$, the tangential component of $J_h^K(n_h)$ is defined as:

\begin{equation}
j_{e_i}  \simeq \mathcal{H}_{e_i}(\bar{D}_n) \dfrac{\mathcal{B}(\delta_i(\varphi / V_{th}))n_{h,k} -  \mathcal{B}(-\delta_i(\varphi / V_{th}))n_{h,j}}{|e_i|}
\end{equation}

where 

\begin{equation}
\delta_i(\varphi / V_{th}) = \dfrac{\varphi_k - \varphi_j}{V_{th}} = 2 \dfrac{(\vect{b}_K\cdot \vect{t}_{e_i}) |e_i|}{2\mathcal{H}_{e_i}(\bar{D}_n) } = 2 \gamma_i
\end{equation}

\begin{equation}
\mathcal{B}(z) = \left\{ \begin{array}{cl}
\dfrac{z}{e^z-1} & z \neq 0
\\
1 & z = 0
\end{array}
\right.
\end{equation}

being $\vect{b}_K$ the relative drift component of the problem on $K$ and $|\gamma_i|$ the P\`eclet number associated with the edge $e_i$.  

\subsubsection{The discretization scheme}

The local contributions to the assembling of the global stiffness matrix and load vector associated with the linear algebraic system read as follows:

\begin{equation}
\Phi_K  = 
{
\tiny 
\left[
\begin{array}{cccc}
- \left( \begin{array}{c}
a_{e12}\mathcal{B}_{12}L^K_{21} + \\
a_{e13}\mathcal{B}_{13}L^K_{31} + \\
a_{e14}\mathcal{B}_{14}L^K_{41}
\end{array} \right)

& a_{e12}\mathcal{B}_{12}L^K_{21} 
& a_{e13}\mathcal{B}_{13}L^K_{31}
& a_{e14}\mathcal{B}_{14}L^K_{41}
\\

%-----------------------------
a_{e21}\mathcal{B}_{21}L^K_{12}
&
- \left( \begin{array}{c}
a_{e21}\mathcal{B}_{21}L^K_{12} + \\
a_{e23}\mathcal{B}_{23}L^K_{32} + \\
a_{e24}\mathcal{B}_{24}L^K_{42}
\end{array} \right)
& a_{e23}\mathcal{B}_{12}L^K_{32}
& a_{e24}\mathcal{B}_{12}L^K_{42}
\\

%-----------------------------
a_{e31}\mathcal{B}_{31}L^K_{31}
& a_{e31}\mathcal{B}_{32}L^K_{32}
&
- \left( \begin{array}{c}
a_{e31}\mathcal{B}_{31}L^K_{31} + \\
a_{e32}\mathcal{B}_{32}L^K_{32} + \\
a_{e34}\mathcal{B}_{34}L^K_{34}
\end{array} \right)

&a_{e34}\mathcal{B}_{34}L^K_{34}
\\

%-----------------------------
a_{e41}\mathcal{B}_{41}L^K_{41}
& a_{e42}\mathcal{B}_{42}L^K_{42}
&a_{e43}\mathcal{B}_{43}L^K_{43}
&
- \left( \begin{array}{c}
a_{e41}\mathcal{B}_{41}L^K_{41} + \\
a_{e42}\mathcal{B}_{42}L^K_{42} + \\
a_{e43}\mathcal{B}_{43}L^K_{43}
\end{array} \right)

\end{array}
\right]
}
\end{equation}

\begin{equation}
A_K = \Phi_K + \dfrac{|K|}{4} diag (\sigma_n)
\end{equation}

\begin{equation}
\vect{F}_K = \dfrac{|K|}{4} (f_1,f_2,f_3,f_4)^T
\end{equation}


\textcolor{red}{Vale la seguente uguaglianza?}

\begin{equation}
L_{ij} = s_{ij} \vect{t}_{ij} \cdot \nabla \psi_i
\end{equation}



\section{Maximum discrete principle}
\textcolor{blue}{Scriviamo qualcosa in merito?Quanto approfondito?}




