\chapter*{Conclusions and future work}
\addcontentsline{toc}{chapter}{Conclusions and future work}

In this MD Thesis, we have addressed the simulation of semiconductor devices in 3D framework. Gummel map algorithm \cite{GummelMap} is employed to solve the Drift-Diffusion model  \cite{Jackson:ElettroClassica}. The Non Linear Poisson has been discretized using the Galerkin finite element method \cite{quarteroni:NumApprox} following a displacement formulation and improving the convergence algorithm with the damping technique \cite{DefulhardDamp}. The Continuity equations have been treated using the EAFE scheme \cite{Zikatanov:EAFE1} reserving a particular attention to the Zikatanov condition (presented in \cite{Zikatanov:EAFE1}) in order to discuss the discrete maximum principle.

This mathematical framework has been implemented through shared libraries using an object-oriented programming language (C++).

The code has been thoroughly tested on different semiconductor devices (p-n junction, p-n junction in oxide and n-channel/p-channel MOSFET), comparing the results with a commercial tool as reference.

\vspace{1cm}

A great effort has been spent in order to compute the current both at contacts and inside the device. In the first case we extended the \textit{residual method} \cite{ContactCurrentRM} to the 3D framework with excellent results.

Concerning the evaluation of the current density we proposed two novel schemes to post-processing the solutions extending the Scharfetter-Gummel scheme to the 3D case:
\begin{itemize}
\item \textit{edge average technique}, based on a primal mixed formulation of the continuity equation, followed by the approximation of the harmonic average of the diffusion coefficient;
\item \textit{alternative upwinding technique}, based on an accurate analysis of the 1D stabilization function associated to the Scharfetter-Gummel technique.   
\end{itemize}

\vspace{1cm}

To conclude, we wish to mention some possible directions for future activities and investigations:
\begin{itemize}
\item the novel schemes presented in Chapter \ref{chap: postprocess} could be also used to discretized the continuity equation. Moreover, as regards to the edge average technique, some cares must be taken due to the highly non linear relation with respect to the quasi Fermi potential levels;

\item in order to complete the integration with the other modules of FEMOS the coupling with the multiphysic environment must be develop;

\item enlarge the available mobility and R/G models (i.e. Philips unified model, Bologna model, Lombardi model);

\item tretment of floating semiconductor regions;

\item development of the mixed hybrid strategy in order to guarantee the conservation of the flux across nodes, edges or faces of the triangulation;

\item implementation of the fully coupled Newton method.
\end{itemize}