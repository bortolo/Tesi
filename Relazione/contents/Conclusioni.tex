\chapter{Conclusions and future work}


In this MD Thesis, we have presented 

We accomplished our goal

A great effort has been spent in order to compute the current both at contacts and inside the device. In the first case we extended to the 3D framework the \textit{residue method} with excellent results over all tests performed. About the current density we investigated the standard approaches using equation  and  . As expected the standard Drift-Diffusion formula gives bad results expecially when the balance between the drift and diffusion component becomes critical. On the other side using the gradient of the quasi fermi potentials ensure good results negletting the contact problems due to the boundary layers. Some modifications has been presented for both methods.
 We presented the extension to the 3D case of the Sharfetter-Gummel formula and we obtained good results
 
 Although the boundary layer problems still remains at contacts.






Despite our best efforts, there are still many issues to be addressed:
\begin{itemize}
\item the conservation of the fluxes between neighbouring elements is a condition 

\item From the mathematical point of view, it would also be interesting to investigated the problems related to the fulfilment of the condition Zikatanov and proposed alternative algorithm in order to ensure the discrete maximum principle.

\end{itemize}




\begin{itemize}
\item Metodo SG vantaggi a livello computazionale, estensione naturale 3D del metodo SG 1D
\item Metodo DD corretto risolve i problemi di contatto, pi\`u soggetto ad instabilit\`a numeriche, rispetta una propriet\`a importante in maniera forte
\item Possiblit\`a di utilizzare questi metodi come nuovo schema di discretizzazione: problemi al contatto per il metodo SG (problema della media armonica con metodo misto).Per quanto riguarda il DD corretto bisogna pensarci.
\end{itemize}