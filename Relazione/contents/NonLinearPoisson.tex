\subsection{Nonlinear Poisson Equation}

Using the definition of the Frech\`et derivative one can calculated the Jacobian matrix of problem \referenzaeq{eq: NLP system}, which in this case has $1\times 1$ dimensions. In contrast with the Fully Coupled Newton Method \referenzaeq{eq: abstract newton's method}, the functional iteration has a unique variable which is $\varphi$:

\begin{equation}
[\varphi] \rightarrow F([\varphi])
\end{equation}

Therfore given $\varphi^0$, the Newton step for the linearized non linear Poisson equation looks as follows (remember that carrier densities are computed with the Maxwell-Boltzmann approximation):

\begin{equation}
\label{eq: newton step NLP}
\begin{cases}

\nabla \cdot (-\epsilon_s \nabla \delta\varphi^k) 
+   \dfrac{1}{V_{th}} \sigma_s^k\delta\varphi^k 
 =  f_s^k & in \psp{3} \Omega_s
  \\
\nabla \cdot (-\epsilon_{ox} \nabla \delta\varphi^k) =  f_{ox}^k & in \psp{3} \Omega / \Omega_s 
\\
\delta \varphi^k = 0 & on \psp{3} \Gamma_D 
\\
\nabla \delta \varphi^k \cdot \vect{n} = 0 & on \psp{3} \Gamma_N
\\
\varphi^{k+1}=\varphi^k+\delta \varphi^k
\end{cases} 
\end{equation}

having set,

\begin{align*}
\sigma_s^k(\varphi^{k},\varphi_n^{i},\varphi_p^{i}) & =q(p(\varphi^k,\varphi_p^i)-n(\varphi^k,\varphi_n^i))
\\
f_s^k(\varphi^k,\varphi_n^i,\varphi_p^i) & = \nabla \cdot (-\epsilon \nabla \varphi^k) + q\left[ p(\varphi^k,\varphi_p^i)-n(\varphi^k,\varphi_n^i)) + N_D^+-N_A^- \right]
\\
f_{ox}(\varphi^k) & = \nabla \cdot (-\epsilon \nabla \varphi^k) 
\end{align*}

We remark the importance to give an appropriate $\varphi^0$ which respects Dirichlet boundary condition $\varphi_D$, although the problem can't satisfied in strong manner such condition. Moreover with suitable intial guess we intend also a function which hopefully is near as possible at the attractive region of the problem \referenza{theorem: newton convergence}.
The first two equations of system \referenzaeq{eq: newton step NLP} constitute a classical DR (Diffusion-Reaction) problem in $\Omega$, respect the variable $\delta \varphi^i$. We are able to consider a unique piecewise electric permettivity, force term and reaction defined as follows:
\begin{align*}
\epsilon & = \epsilon_s \mathcal{I}_{\Omega_s} + \epsilon_{ox} \mathcal{I}_{\Omega / \Omega_s} \\
f & = f_s \mathcal{I}_{\Omega_s} + f_{ox} \mathcal{I}_{\Omega / \Omega_s} \\
\sigma & = \sigma_s \mathcal{I}_{\Omega_s}
\end{align*}

The more generalized form of \referenzaeq{eq: newton step NLP} reads as follows:
\begin{equation}
\label{sys: NLP general problem linearized}
\left\{
\begin{array}{rcll}
\nabla \cdot (-\epsilon \nabla \delta \varphi^k) + \sigma^k \delta \varphi^k & = &  f^k & \psp{15} in \psp{2} \Omega \\
\delta \varphi^k & = & 0 & \psp{15} on \psp{2} \Gamma_D \\
\nabla \delta \varphi ^k\cdot \vect{n} & = & 0 & \psp{15} on \psp{2} \Gamma_N
\\
\varphi^{k+1} & = & \varphi^k + \delta \varphi^k
\end{array}
\right.
\end{equation}

The well-posedness of such problem is ensured by several (and physical) hypotesis:
\begin{itemize}
\item $\epsilon \in L^{\infty}(\Omega)$ and $\exists m$ s.t. $0 < m \leq \epsilon$ (a.e.) in $\Omega$;
\item  $\sigma \in L^{\infty}(\Omega)$ and $\exists m$ s.t. $0 < m \leq \sigma$ (a.e.) in $\Omega_s$.
\end{itemize}

The relative weak formulation reads as follows: given $f \in L^2(\Omega)$ find $\delta\varphi \in H^1_{\Gamma_D}(\Omega)$ such that 

\begin{equation}
\label{eq: NLP weakformulation}
\int_{\Omega} \epsilon \nabla \delta\varphi \nabla v \, d\Omega + \int_{\Omega} \sigma^{(k)}\delta \varphi v \, d\Omega = \int_{\Omega} f^{(k)}v \, d\Omega \psp{15} \forall v \in H^1_{\Gamma_D}(\Omega)
\end{equation}

For the well-posedness of \referenzaeq{eq: NLP weakformulation} is useful define the sequent quntities:
\begin{equation*}
\begin{array}{ll}
\epsilon_M = max_{\Omega} \epsilon & \epsilon_m = min_{\Omega} \epsilon \\
\sigma_M = max_{\Omega} \sigma & \sigma_m = max_{\Omega} \sigma = 0 \\
\end{array}
\end{equation*}
Take into account the above hypotesis one can demonstrate:
\begin{itemize}
\item \textbf{Continuity},
\begin{equation*}
\begin{array}{ll}
\forall u,v \in H^1_{\Gamma_D} &\\ \\
|\int_{\Omega} \epsilon \nabla u \nabla v + \int_{\Omega} \sigma^{(k)}u v| 
& \leq \epsilon_{M} ||\nabla u ||_{L^2} || \nabla v ||_{L^2} +  \sigma_{M} ||u ||_{L^2} ||v ||_{L^2} 
\\
& \leq max\{\epsilon_{M}, \sigma_{M} \}  
\left( ||\nabla u ||_{L^2} || \nabla v ||_{L^2} +   ||u ||_{L^2} ||v ||_{L^2} \right)
\\
& \leq max\{\epsilon_{M}, \sigma_{M} \}  
||u ||_{H^1_{\Gamma_D}} || v ||_{H^1_{\Gamma_D}}
\end{array}
\end{equation*}

\item \textbf{Coercivity,}
\begin{equation*}
\begin{array}{ll}
\forall u \in H^1_{\Gamma_D} &\\ \\
|\int_{\Omega} \epsilon \nabla u \nabla u + \int_{\Omega} \sigma^{(k)}u^2| 
& \geq \epsilon_{m} ||\nabla u ||_{L^2}^2  +  \sigma_{m} ||u ||_{L^2}^2 
\\
& =  \epsilon_{m} ||\nabla u ||_{L^2}^2 
\\
& = \epsilon_{m} |\nabla u |_{H^1_{\Gamma_D}}^2 
\end{array}
\end{equation*}

\item \textbf{Continuity of the functional,}
\begin{equation*}
\begin{array}{ll}
|\int_{\Omega} f^{(k)} v |
& \leq ||f^{(k)} ||_{L^2}||v ||_{L^2} \psp{15} \forall v \in H^1_{\Gamma_D}
\end{array}
\end{equation*}
\end{itemize}

Then we can state that there exists a unique solution of the linearized Poisson equation.