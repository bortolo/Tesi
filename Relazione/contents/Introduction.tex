\chapter{Introduction}

In 1947 John Bardeen, William Shockley and Walter Brattain (three scientists of Bell Telephone Labs) invented the bipolar transistor and since that crucial point there has been a growth  of the semiconductor industry never known before, with serious impact on the way people work and live today. 

Before reach the functionality and the miniaturization of modern devices, some fundamental steps has been made.
In 1958 the first intagrated circuits (IC) was produced, followed by the introduction of the first MOSFET(1960) and CMOS(1963). Into these inventions the first micro-processor(1971) sank his roots  and since that time until present, an ever-increasing progress has continued, according to the indication of \textit{Moore's Law} (formulated by Gordton Moore in 1965).

These events led microelectronic industry at the doors of the VLSI era (Very-Large-Scale-Integration). Indeed in the last thirty years the benefits of miniaturization have been the key in the evolutionary progress leading to today's computers, wireless units, and comunication systems that offer superior performance, dramatically reduced cost per function, and much reduced physical size.

The large worldwide investment in VLSI technology constitutes a formidable driving force that guarantee the continued progress in IC integration density and speed, for as long as physical principles will allow.

From this point we want start and remark that the aim of numerical simulations is the comprehension of the physical phenomenon which lies behind the function of modern device. 

Even if many commercial software are able to resolve different physic situations, they are often specialized on precise physic branch: obviously this strategic choice guarantees more efficiency but it implies a lost in generality. The conseguence is that the work of the model analyst became harder when he have to afford problems located in the middle of different phenomenon. 

Consider, as example, the functionality of a new device, which its electric behaviour is strong influeced by its mechanical response. Basically you are interested to the resolution of Maxwell's law  (which is well performed by SDEVICE simulator) and the Navier-Lam\`e equations (which is well performed by COMSOL simulator). Now the question is: how to put in comunication the different outputs?
Take into account that it's not possbile known precisely how the above programs resolves the equations, which implies a relevant risk when you decide to combine the solutions. 
In other words the development of an own code is at least desirable and possibly helpful: the main advantage is the total control on simulation procedure and the possibility of fully customize, the major drawback is that the improvement of a personal code needs time and human resources, which in many cases are not avaible.   

The FEMOS project (\textit{Finite Element Method Oriented Solver}) tries to overcome the above limitations. 
Even if modern devices present innovative and unexpected behaviour, we can't avoid the treatment of the classical semiconductor devices from the simulation possibilties of FEMOS.
This thesis found its origin in the development of this achievement, but subjects cover a spread wide area in terms of models and kind of devices. Therfore this work intends to effort some specific points in order to give the basic support to improve more models and features. 

In the first chapter we recall briefly the semiconductor material properties providing necessary informations about physic behaviours and relations between the fundamental quantity (e.g. electrostatic potential, electric field, carrier densities and current densities). Finally, we arrive at the classical Drift-Diffusion model for semiconductor and present several physic models about carrier mobility and generation/recombination phenomena.

The second chapter consists of two main section. The first one present the geometry framework where we operate and introduce some useful notations. In the second one is exposed the algorithm used in order to effort the equations treated in the first chapter: we chose a decoupled Gummel map approach.

In the third chapter  are treated the well-posedness analysis and the numerical approximation of the equations.

The fourth chapter contains the results of the performed simulations. The validation tests are accomplished comparing the results obtained with the FEMOS code with a commercial code (SDEVICE). Moreover we treated the delicate issue of the calculation of the current at contacts over several semiconductor devices, extending to the 3D case the \textit{residual method} \textcolor{red}{ref}.

In the fivth chapter

%\begin{itemize}
%\item development of a finite element based simulator for semiconductor devices which deals with multiple generation/recombination and mobility models;
%\item check solutions obtained against commercial software (SDEVICE);
%\item definition and implementation of a new way to compute the current density inside the device;
%\item extension of the residual method presented in \textcolor{red}{referenza} for the 3D case;
%\item evaluation of the possibility to extend the residual method at the computation of the current density inside the device.
%\end{itemize}
