\chapter{Introduction}

In 1947 John Bardeen, William Shockley and Walter Brattain (three scientists of Bell Telephone Labs) invented the bipolar transistor and since that crucial point there has been a growth  of the semiconductor industry never known before, with serious impact on the way people work and live today. 

Before reach the functionality and the miniaturization of modern devices, some fundamental steps has been made.
In 1958 the first intagrated circuits (IC) was produced, followed by the introduction of the first MOSFET(1960) and CMOS(1963). Into these inventions the first micro-processor(1971) sank his roots  and since that time until present, an ever-increasing progress has continued, according to the indication of \textit{Moore's Law} (formulated by Gordton Moore in 1965).

These events led microelectronic industry at the doors of the VLSI era (Very-Large-Scale-Integration). Indeed in the last thirty years the benefits of miniaturization have been the key in the evolutionary progress leading to today's computers, wireless units, and comunication systems that offer superior performance, dramatically reduced cost per function, and much reduced physical size.

The large worldwide investment in VLSI technology constitutes a formidable driving force that guarantee the continued progress in IC integration density and speed, for as long as physical principles will allow.

From this point we want start and remark that the aim of numerical simulations is the comprehension of the physical phenomenon which lies behind the function of modern device. 

Even if many commercial software are able to resolve different physic situations, they are often specialized on particular physic branch: obviously this strategic choice guarantees more efficiency but it implies a lost in generality. The consequence is that the work of device engineer becomes harder when he had to analyze electrical response coming from different phenomena. 

Let us consider, as example, the functionality of a new device, which its electric behaviour is strong influeced by its mechanical response. Basically you are interested to the resolution of Maxwell's law  (which is well performed by SDEVICE simulator) and the Navier-Lam\`e equations (which is well performed by COMSOL simulator). Now the question is: how to put in comunication the different outputs?
Because it's not possbile known precisely how the above programs resolves the equations, a relevant risk occurs when you decide to combine the solutions. 
In other words the development of an own code is at least desirable and possibly helpful: the main advantage is the control on simulation procedure and the possibility of fully customization, the major drawback is that this requires time and human resources, which in many cases are not avaible.   

The FEMOS project (\textit{Finite Element Method Oriented Solver}) tries to overcome the above limitations. FEMOS is designed for the treatment of chemical, mechanical, thermal and fluid phenomena. 
In this project we can't avoid the treatment of the classical semiconductor devices, this thesis found its origin in the development of this achievement.

In the first chapter we briefly recall the semiconductor material properties, physical behaviours and relations between the fundamental quantity (e.g. electrostatic potential, electric field, carrier densities and current densities). The classical Drift-Diffusion model has been detailed discussed; with the needed models for carrier mobility and generation/recombination phenomena.

The second chapter consists of two main sections. The first one presents the geometry framework and introduce some useful notations. The second one illustrates the algorithms used in order to treat the equations of the first chapter (decoupled Gummel map approach).

The third chapter the well-posedness analysis and the numerical approximation of the equations has been detailed discussed.

The fourth chapter contains numerical results of the simulations of typical test case (diode, MOS). The tests are compared with the results of a commercial code (SDEVICE). At the end the calculation of the current at contacts have been performed, extending the \textit{residual method} \cite{ContactCurrentRM} in the 3D case.

The fivth chapter \textcolor{red}{capitolo ancora in fase di costruzione}.

%\begin{itemize}
%\item development of a finite element based simulator for semiconductor devices which deals with multiple generation/recombination and mobility models;
%\item check solutions obtained against commercial software (SDEVICE);
%\item definition and implementation of a new way to compute the current density inside the device;
%\item extension of the residual method presented in \textcolor{red}{referenza} for the 3D case;
%\item evaluation of the possibility to extend the residual method at the computation of the current density inside the device.
%\end{itemize}
