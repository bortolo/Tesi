\chapter{Finite element discretization}

In this section we shall present the classical variational formulation of problems \referenzaeq{eq: newton step NLP linear}, \referenzaeq{eq: LEC system new} and \referenzaeq{eq: LHC system new}. For each kind of PDE problem we give a briefly presentation of the well-posedness analysis. Finally we describe the finite element discretization. 


\section{Non Linear Poisson Equation weak form}

Let us write problem \referenzaeq{eq: newton step NLP linear} in a more compact form

\begin{equation}
\label{sys: NLP general problem linearized}
\left\{
\begin{array}{rcll}
\nabla \cdot (-\epsilon \nabla \delta \varphi^k) + \sigma^k \delta \varphi^k & = &  f^k & \psp{15} in \psp{2} \Omega \\
\delta \varphi^k & = & 0 & \psp{15} on \psp{2} \Gamma_D \\
\nabla \delta \varphi ^k\cdot \vect{n} & = & 0 & \psp{15} on \psp{2} \Gamma_N
\\
\varphi^{k+1} & = & \varphi^k + \delta \varphi^k
\end{array}
\right.
\end{equation}

having set

\begin{align*}
\epsilon & = \epsilon_s \mathcal{I}_{\Omega_{Si}} + \epsilon_{ox} \mathcal{I}_{\Omega_{ox}} \\
f & = f_s \mathcal{I}_{\Omega_{Si}} + f_{ox} \mathcal{I}_{\Omega_{ox}} \\
\sigma & = \sigma_s \mathcal{I}_{\Omega_{Si}}
\end{align*}

where $\mathcal{I}_A(\vect{x})$ is equal to $1$ if $\vect{x}\in A$ and $0$ otherwise.
System \referenzaeq{sys: NLP general problem linearized} constitute a classical Diffusion-Reaction (DR) problem in $\Omega$, respect the variable $\delta \varphi^k$. 
Now we multiply the equation with a test function $v \in H^1_{\Gamma_D}$ and we integrate over all the domain

\begin{equation}
\label{eq: integration first}
- \int_{\Omega} \epsilon \Delta \delta\varphi^k  v \, d\Omega + \int_{\Omega} \sigma^{k}\delta \varphi^k v \, d\Omega = \int_{\Omega} f^{k}v \, d\Omega \psp{15} \forall v \in H^1_{\Gamma_D}(\Omega).
\end{equation}

Applying the Green-formula on \referenzaeq{eq: integration first} and then considering the boundary conditions, we get the weak formulation which reads as: find $\delta \varphi^k \in H^1_{\Gamma_D}(\Omega)$ such that

\begin{equation}
\label{eq: NLP weakformulation}
\int_{\Omega} \epsilon \nabla \delta\varphi^k \nabla v \, d\Omega + \int_{\Omega} \sigma^{k}\delta \varphi^k v \, d\Omega = \int_{\Omega} f^{k}v \, d\Omega \psp{15} \forall v \in H^1_{\Gamma_D}(\Omega)
\end{equation}

We are able to define the following bilinear form

\begin{equation}
\label{eq: bilinear form NLP linearized}
a : H^1_{\Gamma_D}(\Omega) \times H^1_{\Gamma_D}(\Omega) \rightarrow \mathbb{R}, \psp{10} a(u,v) =  \int_{\Omega} \epsilon \nabla u \nabla v \, d\Omega + \int_{\Omega} \sigma^{k}u v \, d\Omega
\end{equation}

and the linear and bounded functional

\begin{equation}
\label{eq: functional NLP linearized}
F:H^1_{\Gamma_D}(\Omega)\rightarrow \mathbb{R}, \psp{10} F(v) = \int_{\Omega} f^{k}v \, d\Omega
\end{equation}

In order to prove the existence and uniqueness of the solution, we apply the \textit{Lax-Millgram theorem} \cite{salsa:EDP} to the weak formulation \referenzaeq{eq: NLP weakformulation}.
The well-posedness is ensured by several and physical hypotesis:

\begin{itemize}
\item $\epsilon \in L^{\infty}(\Omega)$ and $\exists m$ s.t. $0 < m \leq \epsilon$ (a.e.) in $\Omega$;
\item  $\forall k \geq 0$ $\sigma^k \in L^{\infty}(\Omega)$ and $\exists m$ s.t. $0 < m \leq \sigma^k$ (a.e.) in $\Omega_{Si}$.
\end{itemize}

We define some useful quantities:

\begin{equation*}
\begin{array}{ll}
\epsilon_M = max_{\Omega} \epsilon & \epsilon_m = min_{\Omega} \epsilon \\
\sigma_M = max_{\Omega} \sigma & \sigma_m = max_{\Omega} \sigma = 0 \\
\end{array}
\end{equation*}

Take into account the above hypotesis one can demonstrate:
\begin{itemize}
\item \textbf{Continuity of the bilinear form},
\begin{equation*}
\begin{array}{ll}
\forall u,v \in H^1_{\Gamma_D} &\\ \\
|\int_{\Omega} \epsilon \nabla u \nabla v + \int_{\Omega} \sigma^{k}u v| 
& \leq \epsilon_{M} ||\nabla u ||_{L^2} || \nabla v ||_{L^2} +  \sigma_{M} ||u ||_{L^2} ||v ||_{L^2} 
\\
& \leq max\{\epsilon_{M}, \sigma_{M} \}  
\left( ||\nabla u ||_{L^2} || \nabla v ||_{L^2} +   ||u ||_{L^2} ||v ||_{L^2} \right)
\\
& \leq max\{\epsilon_{M}, \sigma_{M} \}  
||u ||_{H^1_{\Gamma_D}} || v ||_{H^1_{\Gamma_D}}
\end{array}
\end{equation*}

\item \textbf{Coercivity of the bilinear form,}
\begin{equation*}
\begin{array}{ll}
\forall u \in H^1_{\Gamma_D} &\\ \\
|\int_{\Omega} \epsilon \nabla u \nabla u + \int_{\Omega} \sigma^{k}u^2| 
& \geq \epsilon_{m} ||\nabla u ||_{L^2}^2  +  \sigma_{m} ||u ||_{L^2}^2 
\\
& =  \epsilon_{m} ||\nabla u ||_{L^2}^2 
\\
& = \epsilon_{m} |\nabla u |_{H^1_{\Gamma_D}}^2 
\end{array}
\end{equation*}

\item \textbf{Continuity of the functional,}
\begin{equation*}
\begin{array}{ll}
|\int_{\Omega} f^{k} v |
& \leq ||f^{(k)} ||_{L^2}||v ||_{L^2} \psp{15} \forall v \in H^1_{\Gamma_D}
\end{array}
\end{equation*}
\end{itemize}

Then we can state that $\forall k \geq 0$ there exists a unique solution of the linearized Non Linear Poisson equation.

\section{Continuity Equations weak form}

Without loss of generality we can consider only the electron continuity equation. System \referenzaeq{eq: LEC system new} is a classical diffusion-advection-reaction (DAR) problem written in conservative form. With a suitable change of variables we are able to treat these PDE's equations likewise the linearized Non Linear Poisson equation in the previous section. Consider the Slotboom variable \referenzaeq{eq: un slotboom}, we can rewrite system \referenzaeq{eq: LEC system new}

\begin{equation}
\small
\left\{
\begin{array}{rcll}
 \nabla \cdot \left( - q D_n e^{(\varphi^{i}/V_{th})}\nabla u_n \right) + \sigma_n^{i-1} e^{(\varphi^{i}/V_{th})} u_n& = & f^{i-1}  & \psp{15} in \psp{2} \Omega_{Si} \\
u_n & = &  n_D e^{(-\varphi^{i}/V_{th})} & \psp{15} on \psp{2} \Gamma_{D,Si} \\
\nabla u_n \cdot \vect{n} & = & 0 & \psp{15} on \psp{2} \Gamma_{N,Si}
\end{array}
\right.
\end{equation}

We can easily obtain the weak formulation repeating the same steps made for the linearized Non Linear Poisson problem. Therefore the weak formulation of the Electron Continuity equation is: find $u_n \in H^1_{\Gamma_{D,Si}}(\Omega)$ such that:

\begin{equation}
\small
\label{eq: LEC weakformulation}
\int_{\Omega_{Si}}  q D_n e^{(\varphi^{i}/V_{th})} \nabla u_n \nabla v \, d\Omega + \int_{\Omega_{Si}} \sigma_n^{i-1} e^{(\varphi^{i}/V_{th})} u_n v \, d\Omega = \int_{\Omega_{Si}} f^{i-1}v \, d\Omega \psp{10} \forall v \in H^1_{\Gamma_{D,Si}}
\end{equation}


The existence and uniqueness of the unkown variable $u_n$ enures the same properties on $n$, thanks to the univocal relation between $u_n$ and $n$.
We have to make some hypotesis on the coefficients $\forall i\geq 0$:
\begin{itemize}
\item $q D_n e^{(\varphi^{i}/V_{th})} \in L^{\infty}(\Omega_{Si})$ and $\exists m$ s.t. $0 < m \leq q D_n e^{(\varphi^{i}/V_{th})}$ (a.e.) in $\Omega_{Si}$;
\item  $\sigma_n^{i-1} e^{(\varphi^{i}/V_{th})} \in L^{\infty}(\Omega_{Si})$ and $\exists m$ s.t. $0 < m \leq \sigma_n^{i-1} e^{(\varphi^{i}/V_{th})}$ (a.e.) in $\Omega_{Si}$.
\end{itemize}

As we made before, we define the relative bilinear form

\begin{equation}
\label{eq: bilinear form LEC}
a(u,v) =  \int_{\Omega_{Si}}  q D_n e^{(\varphi^{i}/V_{th})} \nabla u_n \nabla v \, d\Omega + \int_{\Omega_{Si}} \sigma_n^{i-1} e^{(\varphi^{i}/V_{th})} u_n v \, d\Omega
\end{equation}

and the linear and bounded functional

\begin{equation}
\label{eq: functional LEC}
F(v) = \int_{\Omega_{Si}} f^{i-1}v \, d\Omega
\end{equation}

Now the well-posedness of this problem is ensured repeating the same passages showed in the previous section.



\section{Numerical approximation}



In this section we shall introduce briefly the classical Galerkin's method for the approximation of a weak formulation on $\Omega$. Every weak formulation could be represented in a more compact and generic form as, find $u \in V$ such that

\begin{equation}
a(u,v) = F(v) \psp{10} \forall v \in V
\end{equation}

where $V$ is the space of admissible functions, e.g. $H^1_{\Gamma_D}(\Omega)$, $H^1_{\Gamma_{D,Si}}(\Omega_{Si})$.
 Let us introduce $V_h$ which it is a family of finite-dimensional subspace of $V$, depending by a positive parameter $h$, such that

\begin{equation}
V_h \subset V, \psp{5} dim V_h < \infty \, \forall h>0
\end{equation}

The \textit{Galerkin's problem} reads as, find $u_h\in V_h$ such that:

\begin{equation}
\label{eq: general galerkin problem}
 a(u_h,v_h) = F(v_h) \psp{3} \forall v_h \in V_h
\end{equation}

Let be $\mathcal{T}_h$, a finite partition of $\Omega$, and $K$ a generic element of $\mathcal{T}_h$ such that  $\bar{\Omega} =  \bigcup \bar{K}$. In this case the parameter $h$ refers to the characteristic dimension of the elements $K$.
Let us introduce the general finite element spaces of the polynomial element-wise defined functions:

\begin{equation}
X^r_h(\Omega) := \{v_h \in C^0(\bar{\Omega}): v_h|_K\in \mathbb{P}_r,\forall K \in \mathcal{T}_h \}
\end{equation}

and the relative space where functions vanish on boundaries

\begin{equation}
X^r_{h,\Gamma_D}(\Omega)  := \{ v_h \in X^r_h: v_h|_{\Gamma_D} = 0 \} .
\end{equation}

If $\Omega \in \mathbb{R}^3$ we have:
\begin{equation}
\small
\label{eq: dimensione polinomi}
dim\mathbb{P}_r := \dfrac{(r+1)^3}{2} + \dfrac{(r+1)^2}{2} + \dfrac{r(r+1)(2r+1)}{12} - \dfrac{r(r+1)^2}{2} - \dfrac{r(r+1)}{4}
\end{equation}

More precisely we approximate $H^1_{\Gamma_D}(\Omega)$ with $X^1_{h,\Gamma_D}(\Omega)$ and $H^1_{\Gamma_{D,Si}}(\Omega_{Si})$ with $X^1_{h,\Gamma_{D,Si}}(\Omega_{Si})$. Therefore accordingly with \referenzaeq{eq: dimensione polinomi} we have

\begin{align*}
dim \mathbb{P}_1 & = 4 \\
dim X^1_h & = N_h \\
dim X^1_{h,\Gamma_D} & = N_h - N_g
\end{align*}

where $N_h$ is the number of verteces of the partition $\mathcal{T}_h$ and $N_g$ are the number of verteces lie on Dirichlet boundaries.

We denote by $\{ \psi_j \}_{j=1}^{N_h} $ the Lagrangian basis of the space $X^1_{h}$. Naturally  as $u_h \in X^1_{h}$ there are $u_j \in \mathbb{R}$ with $j=1,\, ... \,, N_h$ such that:


\begin{equation}
u_h = \sum_{j=1}^{N_h} u_j \psi_j
\end{equation}

Because every functions of $V_h$ is a linear combination of $\psi_i$, we can test equation \referenzaeq{eq: general galerkin problem} only for each basis function rather than $\forall v_h \in V_h$. The result of the complete discretization is find $u_j$, with $j = 1, \, . \, . \, . \, , \, N_h$ such that:

\begin{equation}
\sum_{j=1}^{N_h}u_ja(\psi_j,\psi_i) = F(\psi_i) \psp{10} \forall i = 1, \, . \, . \, . \, , \, N_h
\end{equation}

In order to implement this routine it's useful make explicit the subdivsion of the bilinear form on the element of the partition $\mathcal{T}_h$:

\begin{equation}
\sum_{j=1}^{N_h}u_j\sum_{K\in \mathcal{T}_h}a_K(\psi_j,\psi_i) = \sum_{K\in \mathcal{T}_h}F_K(\psi_i) \psp{10} \forall i = 1, \, . \, . \, . \, , \, N_h
\end{equation}


\subsection{Geometrical discretization}

We set every elements $K\in \mathcal{T}_h$ as a tetrahedral of volume $|K|$; we suppose that there exists a constant $\delta>0$ such that:

\begin{equation}
\label{eq: mesh regular condition}
\dfrac{h_K}{\rho_K} \leq \delta \psp{15} \forall K \in \mathcal{T}_h
\end{equation}

where $h_k=diam(K)=max_{x,y\in K}|x-y|$ and $\rho_K$ is the diameter of the sphere inscribed in the tetrahedral $K$. Condition \referenzaeq{eq: mesh regular condition} is the so called \textit{mesh regularity condition} \cite{quarteroni:modnum} and it ensures an istropic partition.
We denote with $\mathcal{E}_h$, $\mathcal{V}_h$ and $\mathcal{F}_h$ the set of all the edges, verteces and faces  
of $\mathcal{T}_h$ respectively, and for each $K\in \mathcal{T}_h$ we denote by $\partial K$ and $\vect{n}_{\partial K}$ the boundary of the element and its outward unit normal.
  
 
 \subsection{Linearized Non Linear Poisson equation}

As regards the linearized NLP equation we have:
\begin{equation}
a(\psi_j,\psi_i)  = \int_{\Omega} \epsilon \nabla \psi_j \nabla \psi_i \, d\Omega + \int_{\Omega} \sigma^{k}\psi_j \psi_i \, d\Omega 
\end{equation}
and the relative restriction on the element is
\begin{equation}
\label{eq: bilinear local discrete}
a_K(\psi_j,\psi_i)  = \int_{K} \epsilon \nabla \psi_j \nabla \psi_i \, dK + \int_{K} \sigma^{k}\psi_j \psi_i \, dK
\end{equation}

Equation \referenzaeq{eq: bilinear local discrete} it's formed by two distinct contributions, the former indentifies the diffusive contribution and generates the so called \textit{stifness matrix}, while the latter refers to the reaction and generates the \textit{mass matrix}.

The coefficient $\epsilon$ is a piece-wise costant function, which changes on different material regions. $\mathcal{T}_h$ is built such that every $K$ belongs to a single region, while it is possible that verteces belong to different regions. Therefore $\epsilon$ is costant over each elements and integral in \referenzaeq{eq: bilinear local discrete} become easier.

As a consequence of choose the discrete space $X^1_{h}$,  we can't expect a better priori estimation error on the solution, than the first order respect the characteristic discretization step $h_K$ \textcolor{red}{referenza}. This implies that is not necessary and useful the use of an high order quadrature, and the trapezoidal rule is enough accurate. 
The main conseguence of using trapezoidal quadrature rule is that extra-diagonal elements of the mass-matrix disappear.
This technique is well known as \textit{lumping procedure} applied on the mass-matrix \textcolor{red}{referenza}.

%\vspace{1cm}
%
%It's possible adopt some simplifications which makes easier the treatment of these integrals. First of all consider that the diffusive coefficient $\epsilon$ can be a rapidly varying function and one could choose to integrate it with a consequently grow up of the computational cost, or it's possible use a suitable average on each elment of this quantity. In view of further discussions of this issue, for each set $S \subset \Omega$ having measure $|S|$, we introduce the following averages of a given function $g$ that is integrable on $S$:
%
%\begin{equation*}
%\mathcal{M}_S(g) = \dfrac{\int_S g \, dS}{|S|} \psp{15} \mathcal{H}_S = (\mathcal{M}_S(g^{-1}))^{-1} 
%\end{equation*}
%
%Notice that $\mathcal{M}_S$ is the usual average, while $\mathcal{H}_S$ is the \textit{harmonic average}. It is well-known that the use of tha harmonic average provides a superior approximation performance than the usual average \textcolor{red}{referenza}.

Finally we obtain the contributions of the local system matrix $A_K^k$:

\begin{equation}
[A_K^k]_{ij}  = \epsilon_K
L_{ij}
+
\dfrac{|K|}{4} \sigma^k_i
\end{equation}

having set

\begin{equation}
\begin{array}{rcl}
L_{ij} & \simeq & \int_K \nabla \psi_i  \nabla \psi_j \, d\Omega \\ \\
\sigma^k_i & =  &\sigma^k(\vect{x}_i)
\end{array}
\end{equation}

Here follows the construction of the right hand side, based on the local contribution approximated with trapezoidal rule:

\begin{equation}
[F_K]_i^k =  f^k_i |K| / 4 \simeq \int_{\Omega} f^k \psi_i \, d\Omega 
\end{equation}

The local contributions of each element $K$ can be assembled in the global matrix $A$: let $I$ be the global index of a generic vertex belonging to the partition $\mathcal{T}_h$, we denote by $\mathcal{J}_K: \mathcal{V}_{\mathcal{T}_h} \rightarrow \mathcal{V}_{K}$ the map which connects $I$ to its corresponding local index $i=1, \, . \, . \, . \, , 4$ in the element $K$. Then we have 

\begin{equation}
A_{IJ}^k = \sum_{\substack{\forall K \in \mathcal{T}_h s.t. \\ \mathcal{J}_K(I),\mathcal{J}_K(J) \subset \mathcal{V}_K}}
 [A_K]_{ij}^k
\end{equation}

analogously for the force term $\vect{b}^{(k)}$:

\begin{equation}
b_{I}^k = \sum_{\substack{\forall K \in \mathcal{T}_h s.t. \\ \mathcal{J}_K(I) \subset \mathcal{V}_K}}
 [F_K]_{i}^k
\end{equation}

Once we have built the global matrix $A^k$ and the global vector $\vect{b}^k$ we need to take into account the essential boundary conditions. In fact the displacement formulation is a primal formulation which forces Dirichlet boundary condition in a strong way. Therefore we have to modify the algebraic system. We choose the \textit{diagonalization} techinique which does not alter the matrix pattern nor introduce ill-conditioning for the system.  Let $i_D$ be the generic index of a Dirichlet node, we denote by $[\delta \varphi_{D}]_i$ (which in this case is equal to zero) the known value of the solution $\delta \varphi $ at the node. We consider the Dirichlet condition as an equation of the form $a [\delta \varphi]_i = a [\delta \varphi_{D}]_i$, where $a\neq 0$ is a suitable coefficient. In order to avoid degrading of the global matrix condition number, we take $a$ equal to the diagonal element of the matrix at the row  $i_D$.

Finally we have completed the discretization of (Step 1), which reads as follows:

\begin{equation}
\label{eq: NLP discretizated}
\left\{
\begin{array}{rcl}
A^k\vect{\delta \varphi}^k & = & \vect{b}^k \\
\vect{\varphi}^{k+1} & = & \vect{\varphi}^k +  \vect{\delta \varphi}^k 
\end{array}
\right.
\end{equation}

As every iteration procedure, problem \referenzaeq{eq: NLP discretizated} needs a suitable convergence break criterion. A good method is based on cheking the satisfaction of the fixed point equation \referenzaeq{eq: abstract problem fully} by the $k$-th solution, then the inner loop of the Gummel Map reads as: given a tollerance $toll>0$ solve problem \referenzaeq{eq: NLP discretizated} untill:
\begin{equation}
||\vect{b}(\varphi^{k+1})||_2 > toll
\end{equation}

where $||\cdot ||_2$ is the usual Euclideian norm for a vector.
\subsubsection{Damping}

Nevertheless the theorem \referenzaeq{theorem: newton convergence}, the system \referenzaeq{eq: NLP discretizated} may be affected by difficulties on the convergence velocity.
The main problem associated with the classical Newton method is the tendency to overestimate the length of the present correction step. This phenomenon is frequently termed as \textit{overshoot}. In the case of the semiconductor equations this overshoot problem has often been treated by simply limiting the size of the correction vector ($\delta \varphi$) determined by Newton's method. The usual established modifications to avoid overshoot are given by the follow formulations:


\begin{align}
\tilde{A}(\varphi_k)&=\dfrac{1}{t_k}A(\varphi_k) \label{eq: NLP mod used}
\end{align}

$t_k$ is a properly chosen positive parameter: for $t_k=1$ the modified Newton methods reduce to the classical Newton method. We have now to deal with the question how to choose $t_k$ or $s_k$ such that the modified Newton methods exhibit superior convergence properties compared to the classical Newton method.
For the case \referenzaeq{eq: NLP mod used} there's a simple criterion suggested by Deuflhard \cite{selberherr:SimSem} $t_k$ is taken from the interval $(0,1]$ in such a manner that for any norm,
\begin{equation}
\label{eq: extended criterion}
||A(\varphi_k)^{-1}\vect{b}(\varphi_k-t_kA(\varphi_k)^{-1}\vect{b}(\varphi_k))||<||A(\varphi_k)^{-1}\vect{b}(\varphi_k)||
\end{equation}

Condition \referenzaeq{eq: extended criterion} guarantees that the correction of the k-th iterate is an improved approximation to the final solution, in other words the residual norm can only descents.
This condition can be easily evaluated only if the Jacobian matrix is factored into triangular matrices because the evaluation of the argument of the norm on the left hand side of \referenzaeq{eq: extended criterion} is then reduced to a forward and backward substitution and the evaluation of $\vect{b}(\varphi)$. Although we use an iterative methods (BCG solver) which implies serious diffuclties to the application of the above criterion. Another valid possibility is to use the main diagonal of $A(\varphi_k)$, denoted as $D(\varphi_k)$:
\begin{equation}
\label{eq: easy criterion}
||D(\varphi_k)^{-1}\vect{b}(\varphi_k-t_kD(\varphi_k)^{-1}\vect{b}(\varphi_k))||<||D(\varphi_k)^{-1}\vect{b}(\varphi_k)||
\end{equation}

This is the criterion developed in our code. However the value to use for $t_k$ is a question of trial and error. Frequently one chooses the following sequences:

\begin{align}
t_k & = \dfrac{1}{2^i} \\
t_k & = \dfrac{1}{2^{\dfrac{i(i+1)}{2}}}  
\end{align}

obiuvsly $i$ is the subiterations of damping reached when satisfied \referenzaeq{eq: easy criterion}. Sufficiently close to the solution \referenzaeq{eq: extended criterion} (and so \referenzaeq{eq: easy criterion}) will be satisfied with $t_k=1$ so that the convergence properties of the classical Newton method are recovered.
 


 \subsection{Continuity equations}

Let us write the discretized electron continuity equation in a suitable generic form on the $K$-th element:
\begin{equation}
\label{eq: LEC discretized general}
\left\{
\begin{array}{rcl}
a_h^K(n_h,v_h) & = & \int_{K} J_h^K(n_h) \nabla v_h \, dK + \int_{K} \sigma n_h v_h \, dK 
\\
\\
F(v_h)^K & = & \int_K f v_h \, dK
\\
\\
J_h^K & = & D_K(\bar{D}_n) \nabla  (e^{(-\varphi / V_{th})}n_h)
\end{array}
\right.
\end{equation}

Notice that we came back on carrier density variable, because 
\textcolor{red}{finire spiegazione sulla difficolta numerica variabili di slotboom}

We want to extend now the discussion about the treatment of the diffusive coefficient. The matrix $D_K(\bar{D}_n) \in \mathbb{R}^3$ can be characerized in three different ways:

\begin{equation}
D_K(\bar{D}_n) = \left\{ 
\begin{array}{l}
\mathcal{M}_K(\bar{D}_n) \, \mathcal{I}
\\
\\
\mathcal{H}_K(\bar{D}_n) \, \mathcal{I}
\\
\\
 \dfrac{1}{|K|}\sum_{i=1}^6 \mathcal{H}_{e_i}(\bar{D}_n) |e_i| s_i \vect{t}_i \vect{t}_i^T
\end{array}
\right.
\end{equation} 

These different approaches in the computation of the average of the diffusion coefficient are responsible for the quite different numerical perfomance of the relative methods.
We already presented the standard average and the armonic average and we discussed briefly the advantages of them. 
Now we are interested in the latter method which  is an exponentially fitted treatment of $\bar{D}_n$ over the entire subdomain $K$. 
From \referenzaeq{eq: LEC discretized general} we immeditaly obtain:

\begin{equation}
\label{eq: exp fitted flux}
\vect{J}_h^K = \dfrac{1}{|K|}\sum_{i=1}^6 |e_i| s_i j_{e_i} \vect{t}_i 
\end{equation}

Having defined the flux vector over $K$ in the form \referenzaeq{eq: exp fitted flux}, it is clearly possible to construct a fimily of Galerkin finite element approximations for the continuity equations by a proper choice of the quantities $j_{e_i}$. 

\subsubsection{Edge flux density computation}

We introduce an approximate flux density computation along the edges of $\partial K$ based on the classical one dimensional \textit{Sharfetter-Gummel} method \textcolor{red}{referenza o anticipazione?}. For each edge $e_i$, the tangential component of $J_h^K(n_h)$ is defined as:

\begin{equation}
j_{e_i}  \simeq \mathcal{H}_{e_i}(\bar{D}_n) \dfrac{\mathcal{B}(\delta_i(\varphi / V_{th}))n_{h,k} -  \mathcal{B}(-\delta_i(\varphi / V_{th}))n_{h,j}}{|e_i|}
\end{equation}

where 

\begin{equation}
\delta_i(\varphi / V_{th}) = \dfrac{\varphi_k - \varphi_j}{V_{th}} = 2 \dfrac{(\vect{b}_K\cdot \vect{t}_{e_i}) |e_i|}{2\mathcal{H}_{e_i}(\bar{D}_n) } = 2 \gamma_i
\end{equation}

\begin{equation}
\mathcal{B}(z) = \left\{ \begin{array}{cl}
\dfrac{z}{e^z-1} & z \neq 0
\\
1 & z = 0
\end{array}
\right.
\end{equation}

being $\vect{b}_K$ the relative drift component of the problem on $K$ and $|\gamma_i|$ the P\`eclet number associated with the edge $e_i$.  

\subsubsection{The discretization scheme}

The local contributions to the assembling of the global stiffness matrix and load vector associated with the linear algebraic system read as follows:

\begin{equation}
\Phi_K  = 
{
\tiny 
\left[
\begin{array}{cccc}
- \left( \begin{array}{c}
a_{e12}\mathcal{B}_{12}L^K_{21} + \\
a_{e13}\mathcal{B}_{13}L^K_{31} + \\
a_{e14}\mathcal{B}_{14}L^K_{41}
\end{array} \right)

& a_{e12}\mathcal{B}_{12}L^K_{21} 
& a_{e13}\mathcal{B}_{13}L^K_{31}
& a_{e14}\mathcal{B}_{14}L^K_{41}
\\

%-----------------------------
a_{e21}\mathcal{B}_{21}L^K_{12}
&
- \left( \begin{array}{c}
a_{e21}\mathcal{B}_{21}L^K_{12} + \\
a_{e23}\mathcal{B}_{23}L^K_{32} + \\
a_{e24}\mathcal{B}_{24}L^K_{42}
\end{array} \right)
& a_{e23}\mathcal{B}_{12}L^K_{32}
& a_{e24}\mathcal{B}_{12}L^K_{42}
\\

%-----------------------------
a_{e31}\mathcal{B}_{31}L^K_{31}
& a_{e31}\mathcal{B}_{32}L^K_{32}
&
- \left( \begin{array}{c}
a_{e31}\mathcal{B}_{31}L^K_{31} + \\
a_{e32}\mathcal{B}_{32}L^K_{32} + \\
a_{e34}\mathcal{B}_{34}L^K_{34}
\end{array} \right)

&a_{e34}\mathcal{B}_{34}L^K_{34}
\\

%-----------------------------
a_{e41}\mathcal{B}_{41}L^K_{41}
& a_{e42}\mathcal{B}_{42}L^K_{42}
&a_{e43}\mathcal{B}_{43}L^K_{43}
&
- \left( \begin{array}{c}
a_{e41}\mathcal{B}_{41}L^K_{41} + \\
a_{e42}\mathcal{B}_{42}L^K_{42} + \\
a_{e43}\mathcal{B}_{43}L^K_{43}
\end{array} \right)

\end{array}
\right]
}
\end{equation}

\begin{equation}
A_K = \Phi_K + \dfrac{|K|}{4} diag (\sigma_n)
\end{equation}

\begin{equation}
\vect{F}_K = \dfrac{|K|}{4} (f_1,f_2,f_3,f_4)^T
\end{equation}


\textcolor{red}{Vale la seguente uguaglianza?}

\begin{equation}
L_{ij} = s_{ij} \vect{t}_{ij} \cdot \nabla \psi_i
\end{equation}