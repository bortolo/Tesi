\chapter{Abstract}

Nel presente elaborato vengono illustrati gli strumenti matematici necessari al fine di simulare il comportamento elettrico dei dispositivi a semiconduttore in un contesto 3D. Il modello Drift-Diffusion \cite{Jackson:ElettroClassica} \`e stato scelto come riferimento e la sua risoluzione si \`e basata sull'algoritmo della mappa di Gummel \cite{GummelMap}. Il problema non lineare di Poisson \`e stato affrontato con il metodo di Newton. La discretizzazione delle equazioni \`e stata realizzata secondo il metodo di Galerkin agli elementi finiti (FEM) ed in particolare per il problema di Poisson si \`e scelta una formulazione agli spostamenti, mentre per l'equazione di continuit\`a ci si \`e affidati allo schema numerico EAFE presentato in \cite{Zikatanov:EAFE1}.
Sono stati introdotti diversi modelli di mobilit\`a e di genereazione/ricombinazione dei portatori in maniera tale che possano essere scelti dall'utente a run-time.
Il codice prodotto \`e organizzato in diversi moduli che vanno ad integrarsi a FEMOS (Finite Element Method Oriented Solver): si tratta di un solutore multifisico scritto in C++ e raccolto in librerie dinamiche.
I tests sono stati condotti su vari dispositivi a semiconduttore (diodo, n-MOSFET/p-MOSFET) ed i risultati ottenuti sono stati confrontati con un solutore commerciale (SDEVICE).

Una parte fondamentale del lavoro \`e costituita dalle tecniche sviluppate al fine di valutare la corrente ai contatti e all'interno dei dispositivi. Nel primo caso abbiamo esteso il metodo dei residui  presentato in \cite{ContactCurrentRM} ad un framework 3D. Per quanto riguarda il calcolo della densit\`a di corrente, proponiamo due schemi innovativi volti all'estensione al caso 3D della formula 1D di Scharfetter-Gummel \cite{Gummel:SignAnalys}. Infine presentiamo i risultati numerici confrontandoli con SDEVICE.
