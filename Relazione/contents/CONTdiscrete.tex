\subsection{Continuity Equation}

Let us write the discretized electron continuity equation in a suitable generic form on the $K$-th element:
\begin{equation}
\label{eq: LEC discretized general}
\left\{
\begin{array}{rcl}
a_h^K(n_h,v_h) & = & \int_{K} J_h^K(n_h) \nabla v_h \, dK + \int_{K} \sigma n_h v_h \, dK 
\\
\\
F(v_h)^K & = & \int_K f v_h \, dK
\\
\\
J_h^K & = & D_K(\bar{D}_n) \nabla  (e^{(-\varphi / V_{th})}n_h)
\end{array}
\right.
\end{equation}

Notice that we came back on carrier density variable, because 
\textcolor{red}{finire spiegazione sulla difficolta numerica variabili di slotboom}

We want to extend now the discussion about the treatment of the diffusive coefficient. The matrix $D_K(\bar{D}_n) \in \mathbb{R}^3$ can be characerized in three different ways:

\begin{equation}
D_K(\bar{D}_n) = \left\{ 
\begin{array}{l}
\mathcal{M}_K(\bar{D}_n) \, \mathcal{I}
\\
\\
\mathcal{H}_K(\bar{D}_n) \, \mathcal{I}
\\
\\
 \dfrac{1}{|K|}\sum_{i=1}^6 \mathcal{H}_{e_i}(\bar{D}_n) |e_i| s_i \vect{t}_i \vect{t}_i^T
\end{array}
\right.
\end{equation} 

These different approaches in the computation of the average of the diffusion coefficient are responsible for the quite different numerical perfomance of the relative methods.
We already presented the standard average and the armonic average and we discussed briefly the advantages of them. 
Now we are interested in the latter method which  is an exponentially fitted treatment of $\bar{D}_n$ over the entire subdomain $K$. 
From \referenzaeq{eq: LEC discretized general} we immeditaly obtain:

\begin{equation}
\label{eq: exp fitted flux}
\vect{J}_h^K = \dfrac{1}{|K|}\sum_{i=1}^6 |e_i| s_i j_{e_i} \vect{t}_i 
\end{equation}

Having defined the flux vector over $K$ in the form \referenzaeq{eq: exp fitted flux}, it is clearly possible to construct a fimily of Galerkin finite element approximations for the continuity equations by a proper choice of the quantities $j_{e_i}$. 

\subsubsection{Edge flux density computation}

We introduce an approximate flux density computation along the edges of $\partial K$ based on the classical one dimensional \textit{Sharfetter-Gummel} method \textcolor{red}{referenza o anticipazione?}. For each edge $e_i$, the tangential component of $J_h^K(n_h)$ is defined as:

\begin{equation}
j_{e_i}  \simeq \mathcal{H}_{e_i}(\bar{D}_n) \dfrac{\mathcal{B}(\delta_i(\varphi / V_{th}))n_{h,k} -  \mathcal{B}(-\delta_i(\varphi / V_{th}))n_{h,j}}{|e_i|}
\end{equation}

where 

\begin{equation}
\delta_i(\varphi / V_{th}) = \dfrac{\varphi_k - \varphi_j}{V_{th}} = 2 \dfrac{(\vect{b}_K\cdot \vect{t}_{e_i}) |e_i|}{2\mathcal{H}_{e_i}(\bar{D}_n) } = 2 \gamma_i
\end{equation}

\begin{equation}
\mathcal{B}(z) = \left\{ \begin{array}{cl}
\dfrac{z}{e^z-1} & z \neq 0
\\
1 & z = 0
\end{array}
\right.
\end{equation}

being $\vect{b}_K$ the relative drift component of the problem on $K$ and $|\gamma_i|$ the P\`eclet number associated with the edge $e_i$.  

\subsubsection{The discretization scheme}

The local contributions to the assembling of the global stiffness matrix and load vector associated with the linear algebraic system read as follows:

\begin{equation}
\Phi_K  = 
{
\tiny 
\left[
\begin{array}{cccc}
- \left( \begin{array}{c}
a_{e12}\mathcal{B}_{12}L^K_{21} + \\
a_{e13}\mathcal{B}_{13}L^K_{31} + \\
a_{e14}\mathcal{B}_{14}L^K_{41}
\end{array} \right)

& a_{e12}\mathcal{B}_{12}L^K_{21} 
& a_{e13}\mathcal{B}_{13}L^K_{31}
& a_{e14}\mathcal{B}_{14}L^K_{41}
\\

%-----------------------------
a_{e21}\mathcal{B}_{21}L^K_{12}
&
- \left( \begin{array}{c}
a_{e21}\mathcal{B}_{21}L^K_{12} + \\
a_{e23}\mathcal{B}_{23}L^K_{32} + \\
a_{e24}\mathcal{B}_{24}L^K_{42}
\end{array} \right)
& a_{e23}\mathcal{B}_{12}L^K_{32}
& a_{e24}\mathcal{B}_{12}L^K_{42}
\\

%-----------------------------
a_{e31}\mathcal{B}_{31}L^K_{31}
& a_{e31}\mathcal{B}_{32}L^K_{32}
&
- \left( \begin{array}{c}
a_{e31}\mathcal{B}_{31}L^K_{31} + \\
a_{e32}\mathcal{B}_{32}L^K_{32} + \\
a_{e34}\mathcal{B}_{34}L^K_{34}
\end{array} \right)

&a_{e34}\mathcal{B}_{34}L^K_{34}
\\

%-----------------------------
a_{e41}\mathcal{B}_{41}L^K_{41}
& a_{e42}\mathcal{B}_{42}L^K_{42}
&a_{e43}\mathcal{B}_{43}L^K_{43}
&
- \left( \begin{array}{c}
a_{e41}\mathcal{B}_{41}L^K_{41} + \\
a_{e42}\mathcal{B}_{42}L^K_{42} + \\
a_{e43}\mathcal{B}_{43}L^K_{43}
\end{array} \right)

\end{array}
\right]
}
\end{equation}

\begin{equation}
A_K = \Phi_K + \dfrac{|K|}{4} diag (\sigma_n)
\end{equation}

\begin{equation}
\vect{F}_K = \dfrac{|K|}{4} (f_1,f_2,f_3,f_4)^T
\end{equation}


\textcolor{red}{Vale la seguente uguaglianza?}

\begin{equation}
L_{ij} = s_{ij} \vect{t}_{ij} \cdot \nabla \psi_i
\end{equation}