\chapter{Semiconductor model}

\section{Basic Device Physics}
In these section we present a summary about basic physics properties of semiconductor material accordingly with the quantum mechanics theory \textcolor{red}{referenza per stato solido?}. Moreover a short reconstruction of the Drift-Diffusion model is presented.

\subsection{Intrinsic semiconductor}
In a silicon crystal each atom has four valence electrons to share with its four nearest neighboring atoms. The valence electrons are shared in a paired configuration called a covalent bond. The most important result of the application of quantum mechanics to the description of electrons in a solid is that the allowed energy levels of electrons are grouped into bands. The bands are separated by regions of energy that the electrons in the solid cannot possess: forbidden gaps. The highest energy band that is completely filled by electron at 0[K] is called the \textit{valence band}. The next higher energy band, separated by a frobidden gap from the valence band, is called the \textit{conduction band}.

What sets a semiconductor such as silicon apart from a metal or an insulator is that at absolute zero temperature, the valence band is comletely filled with electrons, while the conduction band is completely empty and the \textit{bandgap} is on the order of 1[eV].
This implies that at room temperature a small fraction of the electrons are excited into the conduction band, leaving behind vacancies (called \textit{holes}) in the valence band.
In contrast, an insulator has a much larger forbidden gap making room-temperature conduction virtually impossible while metals have partially filled conduction bands even at absolute zero temperature, this make them good conductors at any temperature.

In every cases the energy distribution of electrons in a solid is governed by the laws of Fermi-Dirac statistics. For a system in thermal equilibrium, the principal result of these statistics is the \textit{Fermi-Dirac distribution function}, which gives the probability that an electronic state at energy E is occupied by an electron,
\begin{equation}
\label{eq: fermi dirac distribution}
f_D(E) = \dfrac{1}{1+exp\left(\dfrac{E-E_f}{KT}\right)} 
\end{equation}
here $K=1.38\times10^{-23}[J/K]$ is Boltzmann's constant, $T$ is the absolute temperature and $E_f$ is a parameter called the \textit{Fermi level}.

\begin{Definizione}
The Fermi level ($E_f$) is the energy at which the probability of occupation of an energy state by an electron is exactly one-half.
\end{Definizione}

In most cases when the energy is at least several $KT$ above or below the Fermi level \referenzaeq{eq: fermi dirac distribution} can be approximated with the Maxwell-Boltzmann statistics for classical particles, which read as follows:

\begin{equation}
\label{eq: maxwell distribution}
f_D(E)\simeq f_{MB}(E) = 
\begin{cases}
exp\left(-\dfrac{E-E_f}{KT}\right) & E\gg E_f \\
1-exp\left(-\dfrac{E_f-E}{KT}\right) & E \ll E_f
\end{cases}
\end{equation}

Fermi level plays an essential role in characterizing the equilibrium state of a stystem, it is important to keep in mind the sequent observation.

\begin{Osservazione}
When two systems are in thermal equilibrium with no current flow between them, their Fermi levels must be equal, in other words for a continuous region of metals and/or semiconductors in contact, the Fermi level at thermal equilibrium is flat (spatially constant throughout the region).
\end{Osservazione}

What we are interested is a suitable formulation of che carrier concentration which for electrons is given by the sequent integral:
\begin{equation}
\label{eq: carrier densiy integral}
n = \int_{E_c}^\infty N(E)f_D(E) \, dE
\end{equation}

With $N(E)dE$ we indicate the number of electronic states per unit volume with an energy between $E$ and $E+dE$ in the conduction band. In general \referenzaeq{eq: carrier densiy integral} is a Fermi integral of the order $1/2$ and must be evaluated numerically. In order to avoid this complication it's useful consider semiconductor materials such that theirs Fermi levels stay at least $3KT/q$ below the edge of the conduction band (for holes we consider the same approximation above the valence band). This kind of semiconductors are usual known as nondegenerated. The concrete conseguence of this hypotesis is that we can approximate the Fermi-Dirac distribution by the Maxwell-Boltzmann distribution and resolve \referenzaeq{eq: carrier densiy integral} in the analytically way, obtained the sequent results,

\begin{align}
n & = N_c exp\left(-\dfrac{E_c-E_f}{KT}\right) \label{eq: n density fd}\\
p & = N_v exp\left(-\dfrac{E_f-E_v}{KT}\right)  \label{eq: p density fd}
\end{align}

where $N_c$ and $N_v$ are the \textit{effective density of states}.
If we consider an intrisic semiconductor we can say that $n=p$ and obtained the \textit{intrinsic Fermi level} $E_i$ using equations \referenzaeq{eq: n density fd} and \referenzaeq{eq: p density fd}:

\begin{equation}
\label{eq: midgap equilibrium}
E_i=E_f=\dfrac{E_c+E_v}{2} - \dfrac{KT}{2}ln\left(\dfrac{N_c}{N_v}\right)
\end{equation}

Back sostitution of \referenzaeq{eq: midgap equilibrium} in \referenzaeq{eq: n density fd} give us the expression of the intrinsic carrier concentration $n_i=n=p$:


\begin{equation}
\label{eq: ni equilibrium NcNv}
n_i = \sqrt{N_cN_v}exp\left(-\dfrac{E_g}{2KT}\right)
\end{equation}

\begin{Osservazione}
Since the thermal energy, $KT$ is muc smaller than the usual semiconductor bandgap $E_g$, the intrinsic Fermi levelo is very close to the midpoint between the conduction band and the valence band.
\end{Osservazione}

Equations \referenzaeq{eq: n density fd} and \referenzaeq{eq: p density fd} can be rewritten in terms of the intrinsic carrier density ($n_i$) and the midgap energy level ($E_i$) :

\begin{align}
n & = n_i exp\left(\dfrac{E_f-E_i}{KT}\right) \label{eq: n density mb}\\
p & = n_i exp\left(\dfrac{E_i-E_f}{KT}\right)  \label{eq: p density mb}
\end{align}

Finally we remark a fundamental and useful relation, since any change in $E_f$ causes reciprocal changes in $n$ and $p$ the product

\begin{equation}
\label{eq: legge di azione di massa}
np=n_i^2
\end{equation}

in equilibrium is a costant, independent of the Fermi level position.


\subsection{Extrinsic semiconductor}
Intrinsic semiconductor at room temperature has an extremely low free-carrier concentration, therefore, its resistivity is very high. In order to make semiconductor a better conductor it's usual add impurities which introduce additional energy levels in the forbidden gap and can be easily ionized to add either electrons to the conduction band or holes to the valence band. In other words the electrical conductivity of silicon is then dominated by the type and concentration of the impurity atoms, or dopants.

There are some different materials used to construct electric devices but the most known and used is silicon. One of the most important charcateristic of silicon is that it's a column-IV element with four valence electrons per atom. This implies that there are two types of impurities which are electrically active: those from column V such as arsenic or phosphorus, and those from column III such as boron.

A column-V atom in a silicon lattice tends to have one extra electron loosely bonded after forming covalent bonds with other silicon atoms. In most cases, the thermal energy at room temperature is sufficient to ionize the impurity atom and free the exra electron to the conduction band. Such type of impurities are called \textit{donors}; they become positively charged when ionized. Silicon material doped with column-V impurities or donors is called \textit{n-type} silicon.

One the other hand, a column-III impurity atom in a silicon lattice tends to be deficient by one electron when forming covalent bonds with other silicon atoms. Such an impurity atom can also be ionized by accepting an electron from the valence band, which leaves a free-moving hole that contributes to electrical conduction. These impurities are called \textit{acceptors}: they become negatively charged when ionized. Silicon material doped with column-III impurities or acceptors is called \textit{p-type} silicon.

A p-type or an n-type is named as \textit{extrinsic} silicon.
In terms of the energy-band diagrams, donors add allowed electron states in the bandgap close to the conduction-band edge, while acceptors add allowed states just above the valence-band edge.

In contrast to intrinsic silicon, the Fermi level in an extrinsic silicon is not located at the midgap. The Fermi level in n-type silicon moves up towards the conduction band and on the other hand the Fermi level in p-type silicon moves down towards the valence band.
The exact position of the Fermi level depends on both the ionization energy and the concentration of dopants. For example, for an n-type material with a donor impurity concentration $N_d$ the charge neutrality condiction in silicon requires that
\begin{equation}
\label{eq: equilibrium charge in n-type}
n = N_d^+ + p
\end{equation}
 where $N_d^+$ is the density of ionized donors.  Similarly for a p-type material with acceptor impurity concentration $N_a$ we have
\begin{equation}
\label{eq: equilibrium charge in p-type}
p = N_a^- + n
\end{equation}
 
 For the sake of simplicity we consider in this work that at room temperature all impurties are ionized ($N_d = N_d^+$ and $N_a = N_a^-$).
 Substituting \referenzaeq{eq: n density fd} and \referenzaeq{eq: p density fd} in \referenzaeq{eq: equilibrium charge in n-type} and \referenzaeq{eq: equilibrium charge in p-type}, solved the algebraic equation, one obtains
 
 \begin{align}
 E_c-E_f = KTln\left(\dfrac{N_c}{N_d}\right)  \label{eq: Ef in n-type}\\
 E_f-E_v= KTln\left(\dfrac{N_v}{N_a}\right) \label{eq: Ef in p-type}
 \end{align}

Equation \referenzaeq{eq: legge di azione di massa} is independent of the dopant type and Fermi level position and in n-type and p-type respectively we obtain
\begin{equation}
p = \dfrac{n_i^2}{N_d} \psp{10} n = \dfrac{n_i^2}{N_a}
\end{equation}
Instead of using $N_c$, $N_v$ and referring to $E_c$ and $E_v$ equation \referenzaeq{eq: Ef in n-type} and \referenzaeq{eq: Ef in p-type} can be written in a more useful form in terms of $n_i$ and $E_i$ defined by eqautions \referenzaeq{eq: ni equilibrium NcNv} and \referenzaeq{eq: midgap equilibrium}:

 \begin{align}
 E_f-E_i = KTln\left(\dfrac{N_d}{n_i}\right)  \label{eq: Ef in n-type Ei} \\
 E_i-E_f = KTln\left(\dfrac{N_a}{n_i}\right)  \label{eq: Ef in p-type Ei} 
 \end{align}

\begin{Osservazione}
The distance between the Fermi level and the intrinsic Fermi level near the midgap is a logarithmic function of doping concentration.
\end{Osservazione}

This situation causes two important conseguences:
\begin{itemize}
\item non linearity relations between potential and desities,
\item exponential dependence of densities from potential.
\end{itemize}

\subsection{Densities at nonequilibrium}

The above discussion applies only when both the electron end hole densisties take on their local equilibrium values and a local Fermi level can be defined. It is often in VLSI device operation to encounter nonequilibrium sistuations where the densties of one or both types of carriers depart from their equilibrium values given by \referenzaeq{eq: n density mb} and \referenzaeq{eq: p density mb}.
In particular, the minority carrier concentration can be easily overwhelmed by injection from neighboring regions. Under these circumstances, while the electrons are in local equilibrium with themselves and so are the holes, electrons and holes are not in equilibrium with each other. In order to extend the kind of relationship between Fermi level and current densities discussed above, one can introduce separate Fermi lvels for electrons and holes. They are called \textit{quasi Fermi levels} defined as $E_{fn}$ and $E_{fp}$ they replace $E_f$ in \referenzaeq{eq: n density mb} and \referenzaeq{eq: p density mb}:

\begin{align}
n & = n_i exp\left(\dfrac{E_{fn}-E_i}{KT}\right) \label{eq: non eq n density mb}\\
p & = n_i exp\left(\dfrac{E_i-E_{fp}}{KT}\right)  \label{eq: non eq p density mb}
\end{align}

In this regard, quasi Fermi levels have a similar physical interpretation in terms of the state occupancy as the Fermi level.
\begin{Osservazione}
The electron density in the conduction band can be calculated as if the Fermi level is at $E_{fn}$, and the hole density in the valence band ca be calculated as if the Fermi level is ag $£_{fp}$.
\end{Osservazione}

\subsection{Carrier transport in semiconductor}

Carrier transport or current flow in silicon is driven by two different mechanisms:
\begin{itemize}
\item the \textbf{drift} of carriers, which is caused by the presence of an electric field;
\item the \textbf{diffusion} of carriers, which is caused by and electron or hole concentration gradient.
\end{itemize}

\subsubsection{Drif current - Ohm's law}

When an electric field is applied to a conducting medium containing free carriers, the carriers are accelerated and acquire a drift velocity superimposed upon their random thermal motion.

\begin{Osservazione}
The drift velocity of holes is in the direction of the applied field, and the drift velocity of electrons is opposite to the field.
\end{Osservazione}

The velocity of the carriers does not increase indefinitely under field acceleration, since they are scattered frequently and lose their acquired momentum after each collision.
During their motion throughout the lattice struccture, carriers travel at an average speed definded as
\begin{equation}
\vect{v}_d^\eta = \pm \dfrac{q\vect{E}\tau_\eta}{m_\eta}  \psp{20} \eta=\{h,e\}
\end{equation}

where $q=1.602e^{-19}[C]$ is the elementary charge, $\vect{E}$ is the electric field, $\tau_\eta$ is the average time of flight of the carrier between two consecutive interactions with the atoms of the lattice and $m_\eta$ is the effective mass.
The coefficient $q\tau_\eta / m_\eta$ characterizes how quickly a carrier can move through the lattice and it's well known as carrier mobility. It's usual idicate this quantity with the greek letter $\mu$, its standard dimensions are $[m^2V^{-1}s^{-1}]$.
In general, one can use \textit{Matthiessen's rule} to include different contributions to the mobility:
\begin{equation}
\dfrac{1}{\mu} = \dfrac{1}{\mu_L} + \dfrac{1}{\mu_I} + \cdots
\end{equation}
where $\mu_L$ and $\mu_I$ correspond to the lattice and impurity scattering limited components of mobility, for a more detailed description of mobility models see \cite{ModernVLSIdevices}. We chose this approach during the implementation of the code.

Therefore the drift current density for a given electron ($n$) or hole ($h$) concentration, reads as follows:

\begin{align}
\vect{J}_n =& -qn\vect{v}_d^n = qn\mu_n\vect{E}  \label{eq: drift electron}\\ 
\vect{J}_p =& +qp\vect{v}_d^p = qp\mu_p\vect{E} \label{eq: drift hole}
\end{align}

The scalar coefficient $qn\mu_n(qp\mu_p)$ is often summerized by the electron (hole) conductivity $\sigma_n(\sigma_p)$. Now you can rewritten \referenzaeq{eq: drift electron} and \referenzaeq{eq: drift hole} and obtain:

\begin{align}
\vect{J}_n =&\sigma_n\vect{E}  \label{eq: ohm drift electron}\\ 
\vect{J}_p =& \sigma_p\vect{E} \label{eq: ohm drift hole}
\end{align}

Relations \referenzaeq{eq: ohm drift electron} and \referenzaeq{eq: ohm drift hole} expresses the well known \textit{Ohm' law} stating that in the conducting material  the current density is directly proportional to the applied electric field.


\subsubsection{Diffusion current - Fick's law}

In semiconductor devices it's usual have different profiles of dopant in order to allow particular behaviors, this implies a not uniform concentration of carriers which they also diffuse as a result of the concentration gradient. This leads to an additional current contribution in proportion to the concentration gradient, in mathematical terms, diffusion current densities are given accordingly by the \textit{Fick's law} as follows:
\begin{align}
\vect{J}_n = -D_n(-q\nabla n) \\
\vect{J}_p = -D_p(+q\nabla p)
\end{align}

The proportionally constants $D_n$ and $D_p$ are called the electron and hole diffusion coefficients and have units of $[cm^2s^{-1}]$. Physically, both drift and diffusion are closely associated with the random thermal motion of carriers and their collisions with the silicon lattice in thermal equilibrium. A simple relationship between the diffusion coefficient and the mobility is the well known \textit{Einstein relation}:
\begin{equation}
D_\eta = \dfrac{KT}{q}\mu_\eta
\end{equation}

\subsubsection{Drift-Diffusion transport equations}
\label{subsub:driftdiffusion transport}

Finally we can write the constitutive laws for the current density of electrons and holes:

\begin{align}
\vect{J}_n &= qn\mu_n\vect{E} + qD_n\nabla n  \label{eq: drift diff electron}\\ 
\vect{J}_p &= qp\mu_p\vect{E} - qD_p \nabla p \label{eq: drift diff hole}
\end{align}
The total conduction current density is $\vect{J}=\vect{J}_n+\vect{J}_p$.

One interesting thing is that these constitutive laws can be rewritten in two other ways at least. We want remark this because it's useful find various physical explanations of the same phenomenon and moreover these reinterpreations  give different start points for the discrete solver algorithm.

It should be nice connect current density with the quasi Fermi level, this is possible in fact using the relation between electric potential and electric field,
\begin{equation}
\vect{E}  = -\nabla \varphi
\end{equation}

one can write the current densities as:

\begin{align*}
\vect{J}_n &= -qn\mu_n\left(\nabla \varphi - \dfrac{KT}{qn}\nabla n \right)\\ 
\vect{J}_p &= -qp\mu_p\left(\nabla \varphi+ \dfrac{KT}{qp} \nabla p \right)
\end{align*}

Equations \referenzaeq{eq: non eq n density mb} and \referenzaeq{eq: non eq p density mb} can be substituted into the above to yeld:

\begin{align}
\vect{J}_n = -qn\mu_n\nabla \varphi_n \\
\vect{J}_p = -qn\mu_p\nabla \varphi_p 
\end{align}

With these equations we underlying an important aspect which occur in semiconductor material:
\begin{Osservazione}
The current density is proportional to the gradient of the quasi Fermi potential instead of the electric field.
\end{Osservazione}

The third way to represent the current density is based on \textit{Slotboom variables}. In 1973 Jan Slotboom proposed this change in variables for the two-dimensional numercal simulation of a bipolar transistor:

\begin{align}
u_n &= n_iexp\left(-\dfrac{\varphi_n}{V_{th}} \right) \label{eq: un slotboom} \\
u_p &= n_iexp\left(\dfrac{\varphi_p}{V_{th}} \right) \label{eq: up slotboom} 
\end{align}

Using the above equations into \referenzaeq{eq: drift diff electron} and \referenzaeq{eq: drift diff hole} we obtain:

\begin{align}
\vect{J}_n &= qD_n exp\left(\dfrac{\varphi}{V_{th}}\right) \nabla u_n \\
\vect{J}_p &= -qD_p exp\left(-\dfrac{\varphi}{V_{th}}\right)  \nabla u_p 
\end{align}

This interpretation totally change the point of view on current density behavior:

\begin{Osservazione}
It's possible explain the drift-diffusion current density in a semiconductor, with a flux totally diffusive of a new kind of carrier and diffusivity coefficient. 
\end{Osservazione}

The vantages and drawback will be discussed better in chapter ...


\section{Drift Diffusion Model for semiconductor}

Simulations on integrated devices works on several different scale, this is the first point which a developer may consider when start to project a simulator code. Include every phenomenon from tha atomic to the macro scale is not possible. An excellent  trade-off between machine time cost and physical accuracy is presented by the \textit{Drift Diffusion model} (DD), which is by far the most widely used mathematical tool for industrial simulaiton of semiconductor devices. In this section we'll show how is obtained the DD model used for our simulations.

\subsection{Drift Diffusion formulation}
 In order to describe the propagation of any electromagnetic signal in a medium, we have to start from the system of Maxwell equations, which reads as follows:

\begin{equation}
\label{eq: Maxwell system}
\left\{
\begin{array}{rcl}
\nabla \times \vect{H} & = & \vect{J} + \dfrac{\partial \vect{D}}{\partial t} \\ \\
\nabla \times \vect{E} & = & - \dfrac{\partial \vect{B}}{\partial t} \\ \\
\nabla \cdot \vect{D} & = & \rho \\ \\
\nabla \vect{B} &  = & 0
\end{array}
\right.
\end{equation}

We are able to complete the system with the following set of constitutive laws that characterize the electromagnetic properties of the medium:

\begin{equation}
\begin{array}{rcl}
\vect{D} & = & \epsilon \vect{E} \\
\vect{B} & = & \mu_m \vect{H}
\end{array}
\end{equation}

From \referenzaeq{eq: Maxwell system} we elaborate the DD model, through some interesting hypostesis which are:
\begin{itemize}
\item Lorentz-Gauge for the vector potential of $\vect{B}$.
\item Quasi static approximation.
\end{itemize}

The second one is related with the IC component sizes and it is a reasonable hypotesis for our simulations.	
The system obtained after this suitable approximation looks as follows:
\begin{equation}
\left\{
\begin{array}{rcll}
\nabla \cdot (-\epsilon \nabla \varphi) & = & \rho &\psp{15} \textit{Poisson equation}\\ \\
\dfrac{\partial \rho}{\partial t} + \nabla \cdot \vect{J} & = & 0 &\psp{15} \textit{Continuity equation}\\
\end{array}
\right.
\end{equation} 




To close the above system we need to specify the mathematical form of the electric charge density ($\rho$) and the electric conduction current density ($\vect{J}$).

As we introduced in the preview section, devices are usually formed by extrinsic semiconductor and this causes the presence in the lattice of two kind of charge:
\begin{itemize}
\item free charge ($\rho_{free}$) (free electron and holes carriers),
\item fixed charge ($\rho_{fixed}$) (inoized dopant impurities).
\end{itemize} 

\begin{equation}
\label{eq: charge balance}
\rho = \underbrace{q(p-n)}_{\rho_{free}} +\underbrace{q(N_D^+-N_A^-)}_{\rho_{fixed}}
\end{equation}

 Notice that we assume $N_D^+$ and $N_A^-$ time invariant.

 Accordingly with the preview hypotesis and replacing \referenzaeq{eq: charge balance}, \referenzaeq{eq: drift diff electron} and \referenzaeq{eq: drift hole}, we can split the continuity equation into the contribute of electrons and holes, the DD model formulation looks as follows:
 
\begin{equation}
\label{eq: full problem}
\left\{
\begin{array}{rcl}
\nabla \cdot (-\epsilon \nabla \varphi) & = & q(p-n+N_D^+-N_A^-)\\ \\
-q\dfrac{\partial n}{\partial t} + \nabla \cdot ( - q\mu_n n \nabla \varphi + qD_n \nabla n )& = & qR\\ \\
q\dfrac{\partial p}{\partial t} + \nabla \cdot (- q\mu_p p \nabla \varphi - qD_p \nabla p )& = & -qR 
\end{array}
\right.
\end{equation}

The system is an incompletely parbolic initial value/boundary problem in three scalar unkown dependent variables $\varphi(\vect{x},t)$, $n(\vect{x},t)$ and $p(\vect{x},t)$. Notice that the problem is a nonlinearly coupled system of PDE's, because of the presence of the drift terms $n\nabla \varphi$ and $p \nabla 	\varphi$. 

From Maxwell equations we are able to guarantee only that $\vect{J}$ is a solenoidal field, we can't say nothing about the properties of $\vect{J_n}$ and $\vect{J_p}$. For this reason there is a new term in the right hand side. We can interpret $R(\vect{x},t)$ as the net rate of generation and recombination.

We consider also the stationary form for our purpose.

\begin{equation}
\label{eq: stationary problem}
\left\{
\begin{array}{rcl}
\nabla \cdot (-\epsilon \nabla \varphi) & = & q(p-n+N_D^+-N_A^-) \\ \\
\nabla \cdot ( - q\mu_n n \nabla \varphi + qD_n \nabla n )& = & qR \\ \\
\nabla \cdot (- q\mu_p p \nabla \varphi - qD_p \nabla p )& = & -qR
\end{array}
\right.
\end{equation}

\subsection{Generation and Recombination phenomenon}
\label{subsection: RG}

The modelling of $R(\vect{x},t)$ is one of the most important feature which one could take into account, in fact it plays an important role in detemrining the current-voltage characteristic of every device.
 
It's important to keep in mind that electrons and holes are in continuos fluctuation due to their thermal energy, but the macroscopic result of such a process at equilibrium is that the net recombination rate is identically zero at each point and at each time level. Therefore our interest is to analyze the deviations from this condition. In every moment the system try to mantain the equilibrium, so it's important underlying that the response with a recombination event happens in order to neutralize an excess of charge, while generation event are usually due to thermal agitation or an external input source.
The phenomenological model for the net recombination rate $R$ is often given by the sequent formulation:
\begin{equation}
\label{eq: generic RG}
R(n,p) = (pn-n_i^2)F(n,p)
\end{equation}
where $F$ is a function modelling the specific recombination/generation (R/G) event.
In the following we present the classical theory about three kind of contribute. 

\subsubsection{Shockley-Read-Hall recombination}

Electron and hole generation and recombination can take place directly between the valence band and the conduction band, or inderactly via trap centers in the energy gap. The latter category includes Shockley-Read-Hall phenomena (SRH), more precisely SRH rate is a two-particle process which matematically expresses the probality that:
\begin{itemize}
\item[$R_{SRH}$] an electron in the conduction band neutralizes a hole at the valence band through the mediation of an unoccupied trapping level located in the energy gap,
\item[$G_{SRH}$] an electron is emitted from the valence band to the conduction band, through he mediation of an unoccupied trapping level located in the energy gap.
\end{itemize}

The following expression is usually employed for the modulating function $F$:

\begin{equation}
F_{SRH}(n,p) = \dfrac{1}{\tau_n(p+n_i)+\tau_p(n+n_i)}
\end{equation}

the quantiaties $\tau_n$ and $\tau_p$ are called \textit{carrier lifetimes} and are phisically defined as the reciprocals of the capture rates per single carrier associated with the energy trap distribution within the semiconductor energy gap. Their typical order of magnitude lies in the range $10^{-3}\mu s\div 1 \mu s$.


\subsubsection{Auger recombination}

Auger R/G is a three-particle process and take place directly between the valence band and the conduction band. We distinguish four cases which depend to the kind of carriers involved in the phenomena:
\begin{itemize}
\item[$R_{AU}^{2n,1p}$] a high-energy electron in the conduction band moves to the valence band where it neutralizes a hole, transmitting the excess energy to another electron in the conduction band;
\item[$G_{AU}^{2n,1p}$] an electron in the valence band moves to the conduction band by taking the energy from a high energy electron in the conduction band and leaves a hole in the valence band;
\item[$R_{AU}^{2p,1n}$] an electron in the conduction band moves to the valence band where it neutralizes a hole, tranmitting the excess energy to another hole in the valence band;
\item[$G_{AU}^{2p,1n}$] an electron in the valence band moves to the conduction band by taking the energy from a high energy hole in the valence band and leaves a hole in the valence band.
\end{itemize}

The following expression is usually employed for the modulating function $F$:

\begin{equation}
F_{AU}(n,p) = C_nn+C_pp
\end{equation}

where the quantitaties $C_n$ and $C_p$ are the so called  Auger capture coefficients tipically of the order of magnitude of $10^{-25}[cm^6s^{-1}]$.
Note that Auger R/G is relevant only when both carrier densities attain high values.


\subsubsection{Impact ionization}

The impac ionization mechanism is a generation three-particle process and it is dissimilar from the previously phenomenon because we can't express its contribute with a relation like \referenzaeq{eq: generic RG}. The high energy carrier generation is triggered by the presence of very high electric fields. Due to these fields an electron could gains enough energy to excite an electron-hole pair out of a silicon lattice bond. Then the process can be repeated until an avalanche of generated carriers is produced within the region.
There are several different models for the II generation, inside our code we implemented the van Overstraeten - de Man \textcolor{red}{referenza manuale sdevice} model based on the Chynoweth law \textcolor{red}{referenza dentro sdevice}:
\begin{equation}
G_{II}(n,p)= \alpha_n n |\vect{v}_n| + \alpha_p p |\vect{v}_p|
\end{equation}

where:

\begin{equation}
\alpha(E_{ava}) = \gamma exp\left(-\dfrac{\gamma b}{E_{ava}} \right)
\end{equation} 
\begin{equation}
\gamma = \dfrac{tanh\left(\dfrac{\hbar \omega_{op}}{2KT_0} \right) }{tanh\left(\dfrac{\hbar \omega_{op}}{2KT} \right)}
\end{equation}

The factor $\gamma$ with the optical phono energy $\hbar \omega_{op}$ expresses the temperature dependence of the phonon gas against which carriers are accelerated.
$E_{ava}$ is the driving force which takes into account how the electrif field influence the generation event. There are two possibilties to compute this quantity:
\begin{itemize}
\item compute the component of the elctrostatic field in the direction of the current
\begin{equation}
E_{ava}^{n,p} = \dfrac{\vect{E}\cdot\vect{J}_{n,p}}{||\vect{J}_{n,p}||}
\end{equation}
\item consider the module of the quasi fermi gradient
\begin{equation}
E_{ava}^{n,p} = |\nabla \varphi_{n,p}|
\end{equation}
\end{itemize}

\subsection{Mobility models}
