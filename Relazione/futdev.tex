\chapter{Conclusione e sviluppi futuri}
In questo progetto abbiamo implementato una prima versione del metodo di riduzione gerarchica di modello in 3D.
I risultati ottenuti sono in accordo con le aspettative e aiutano lo sviluppo della teoria e, anche se alcune ottimizzazioni sono ancora possibili, 
i tempi di calcolo sono gi\`a buoni.
Per la gestione delle condizioni al bordo laterale abbiamo implementato le basi istruite, ma questa non \`e l'unica
scelta possibile e sicuramente \`e una scelta difficile da generalizzare a geometrie complesse.
Durante l'implementazione non ci siamo focalizzati su alcuni dettagli che non presentano difficolt\`a teoriche o tecniche,
che per\`o andrebbero sistemati, quali la generalit\`a sui diversi tipi di elementi finiti e sulla griglia da usare sulla fibra di supporto,
oppure tutte le possibili condizioni di bordo in inflow o in outflow.

Abbiamo invece parzialmente esplorato due possibili direzioni di sviluppo, tra loro collegate,
\begin{itemize}
 \item estensione del metodo a coefficienti non costanti
 \item estensione del metodo a geometrie differenti.
\end{itemize}
Il primo caso \`e gestito nel codice, anche se non ne abbiamo discusso l'implementazione nella
relazione, \`e infatti presente un metodo \texttt{AddADRproblem} che prende in ingresso delle funzioni e non dei coefficienti costanti.
Tuttavia \`e un metodo molto lento, sono necessarie alcune semplici ottimizzazioni e bisogna testarne il funzionamento
in modo pi\`u intenso dato che per ora \`e stato testato solo con dei coefficienti costanti inseriti all'interno di funzioni.
Nel caso si volesse procedere con test pi\`u complessi \`e presente il tutorial numero sei da modificare ( \texttt{6\_nonconst} ).
 
Nel caso dell'estensione a geometrie pi\`u complesse bisogna prendere in considerazione la
mappa. Essa generer\`a dei coefficienti non costanti: sar\`a dunque necessario perfezionare il metodo descritto
sopra. L'estensione alla mappa presenter\`a due problematiche: la prima \`e che comparir\`a nella forma bilineare dei problemi
1D un termine dove la derivata spaziale \`e sulla funzione test e non sulla base associata alla soluzione, tuttavia, 
avendo scelto di usare il pacchetto \texttt{ETA} per l'assemblaggio, questo non risulta un problema.
La seconda problematica riguarda la scelta delle basi sulla sezione trasversale dove si potrebbero utilizzare i polinomi di Legendre.

Altri aspetti invece sono stati considerati solo teoricamente
\begin{enumerate}
 \item creare una struttura dati adeguata ad ospitare la particolare matrice di sistema
 \item implementare la seconda scelta per l'ordinamento della matrice, presentata 
 nell'introduzione teorica, e testare come cambia la velocit\`a del solutore lineare
 \item parallelizzare il codice: con la struttura a blocchi si aprono molte possibilit\`a
 soprattutto nella parallelizzazione dell'assemblaggio, questo \`e sicuramente il punto pi\`u ampio e pi\`u promettente
 \item implementare il metodo su un tubo a sezione circolare per gestire domini di forma generica con mappe pi\`u semplici
 \item implementare il metodo con la logica della domain decomposition (si veda \cite{perotto:2009}).
\end{enumerate}