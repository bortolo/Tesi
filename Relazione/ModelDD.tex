\section{DD Model}
In this work we deal with mathematical modeling and numerical simulation of different semiconductor devices. 

There are several methods to model integrated devices, this project is based on a semi-classical model, in particular we work with the classical Drift-Diffusion model (DD). Maybe this kind of model is the most used for industrial simulation, due to an excellent trade-off between machine time cost and physical accuracy. Nevertheless to describe the propagation of any electromagnetic signal in a medium, we have to start from the system of Maxwell equations, which reads as follows:

\begin{equation}
\label{eq: Maxwell system}
\left\{
\begin{array}{rcl}
\nabla \times \vect{H} & = & \vect{J} + \dfrac{\partial \vect{D}}{\partial t} \\ \\
\nabla \times \vect{E} & = & - \dfrac{\partial \vect{B}}{\partial t} \\ \\
\nabla \cdot \vect{D} & = & \rho \\ \\
\nabla \vect{B} &  = & 0
\end{array}
\right.
\end{equation}

We are able to complete the system with the following set of constitutive laws that characterize the electromagnetic properties of the medium:

\begin{equation}
\begin{array}{rcl}
\vect{D} & = & \epsilon \vect{E} \\
\vect{B} & = & \mu_m \vect{H}
\end{array}
\end{equation}

From \referenzaeq{eq: Maxwell system} we elaborate the DD model, through some interesting hypostesis which are:
\begin{itemize}
\item Lorentz-Gauge for the vector potential of $\vect{B}$.
\item Quasi static approximation.
\end{itemize}

The second one is related with the IC component sizes and characteristics and it is a reasonable hypotesis for our simulations.	
The system obtained after this suitable approximation looks as follows:
\begin{equation}
\left\{
\begin{array}{rcll}
\nabla \cdot (-\epsilon \nabla \varphi) & = & \rho &\psp{15} \textit{Poisson equation}\\ \\
\dfrac{\partial \rho}{\partial t} + \nabla \cdot \vect{J} & = & 0 &\psp{15} \textit{Continuity equation}\\
\end{array}
\right.
\end{equation} 

To close the above system we need to specify the mathematical form of the electric charge density and the electric conduction current density.

It's well known that intrinsic semiconductor does not appreciably allow current flow, for this reason it's usual introducing impurities (called dopants) in the periodic structure. Dopant impurities are divided into two types: 
\begin{itemize}
\item acceptor type, wich provide positive carriers (holes);
\item donor type, wich provide negative carriers (electron). 
\end{itemize}
It is usual point out acceptor concentration with $N_A$ while donor concentration with $N_D$. However for the electric charge density formulation we are interested only in ionized impurities, thus we obtain the sequent costitutive law:
\begin{equation}
\rho = \underbrace{q(p-n)}_{\rho_{free}} +\underbrace{q(N_D^+-N_A^-)}_{\rho_{fixed}}
\end{equation}

We emphasize the two kind of charge present in the device: $\rho_{fixed}$ related to ionized impurities and $\rho_{free}$ related to free carriers in band ($p$ and $n$ are the concentration of holes and electron respectively). Notice that we assume $N_D^+$ and $N_A^-$ time invariant.

In this work we consider only the transport of these two charge carriers in the device. Consistently with this hypotesis the conduction current density can be written as:
\begin{equation}
\vect{J} = \vect{J_n} + \vect{J_p}
\end{equation}

where $J_n$ and $J_p$ are respectively the electric conduction current density of electrons and holes.
To model charge current flow we consider two principal mechanisms:
\begin{itemize}
\item Diffusion current, according to Fick's law.
\item Drift current, according to Ohm's law.
\end{itemize}
The form of these current densities is expressed by the following relations:
\begin{equation}
\begin{array}{rclcl}
\vect{J_n} & = & \overbrace{q\mu_n n \vect{E}}^{Drift} &+& \overbrace{(- qD_n(- \nabla n))}^{Diffusion} \\ \\
\vect{J_p} & = & q\mu_p p \vect{E} &+& (-qD_p \nabla p) 
\end{array}
\end{equation}



 According with the preview hypotesis and replacing the costitutive laws, we obtain the seguent DD model forumulation:
\begin{equation}
\label{eq: full problem}
\left\{
\begin{array}{rcll}
\nabla \cdot (-\epsilon \nabla \varphi) & = & q(p-n+N_D^+-N_A^-)  &\psp{15} \textit{Poisson equation}\\ \\
-q\dfrac{\partial n}{\partial t} + \nabla \cdot ( - q\mu_n n \nabla \varphi + qD_n \nabla n )& = & qR &\psp{15} \textit{Electron Continuity equaiton}\\ \\
q\dfrac{\partial p}{\partial t} + \nabla \cdot (- q\mu_p p \nabla \varphi - qD_p \nabla p )& = & -qR &\psp{15} \textit{Hole Continuity equaiton}
\end{array}
\right.
\end{equation}

The system is an incompletely parbolic initial value/boundary problem in three scalar unkown dependent variables $\varphi(\vect{x},t)$, $n(\vect{x},t)$ and $p(\vect{x},t)$. Notice that the problem is a nonlinearly coupled system of PDE's, because of the presence of the drift terms $n\nabla \varphi$ and $p \nabla 	\varphi$. 

From Maxwell equations we are able to guarantee only that $\vect{J}$ is a solenoidal field, we can't say nothing about the properties of $\vect{J_n}$ and $\vect{J_p}$. For this reason there is a new term in the right hand side. We can interpret $R(\vect{x},t)$ as the net rate of generation and recombination.

We consider also the stationary form for our purpose.

\begin{equation}
\left\{
\begin{array}{rcl}
\nabla \cdot (-\epsilon \nabla \varphi) & = & q(p-n+N_D^+-N_A^-) \\ \\
\nabla \cdot ( - q\mu_n n \nabla \varphi + qD_n \nabla n )& = & qR \\ \\
\nabla \cdot (- q\mu_p p \nabla \varphi - qD_p \nabla p )& = & -qR
\end{array}
\right.
\end{equation}