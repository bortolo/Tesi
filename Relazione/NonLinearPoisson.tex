\section{Nonlinear Poisson Equation}
In this section we'll show how the NLP is resolved in the code. Many decisions have been taken on the management of the interface. Note that the electrostatic problem must be resolved on the whole domain and the right hand side changes from region to region. More precisely we distinguish semiconductor regions from other regions (oxide,metal,etc...). The linearized non linear poisson equation looks as follows:


\begin{equation}
\begin{cases}

\nabla \cdot (-\epsilon \nabla \delta\varphi^{i}) 
+   \dfrac{1}{V_{th}} \sigma\delta\varphi^{i} 
 =  f_s & in \psp{3} \Omega_s
  \\
\nabla \cdot (-\epsilon \nabla \delta\varphi^i) =  f_o & in \psp{3} \Omega / \Omega_s
\end{cases} 
\end{equation}

where we denotes,

\begin{align*}
\sigma(\varphi^{i},\varphi_n^{k},\varphi_p^{k}) & =q(p(\varphi^{i},\varphi_p^{k})-n(\varphi^{i},\varphi_n^{k}))
\\
f_s(\varphi^{i},\varphi_n^{k},\varphi_p^{k}) & = \nabla \cdot (-\epsilon \nabla \varphi^{i}) + q\left[ p(\varphi^{i},\varphi_p^{k})-n(\varphi^{i},\varphi_n^{k})) + N_D^+-N_A^- \right]
\\
f_o(\varphi^{i},\varphi_n^{k},\varphi_p^{k}) & = \nabla \cdot (-\epsilon \nabla \varphi^{i}) 
\end{align*}


Now we are able to proceed with the classical theory for DAR problem. We present a brief reminder of the weak formulation of  


\textcolor{blue}{Qui dipende da come vogliamo introdurre FEMOS...sarebbe carino far capire la scelta che è stata fatta di porre nei nodi di frontiera del silicio il valore della forzante e della reazione del silicio.Ma ovviamente questo discorso necessita una introduzione sui casi test.}

\subsection{Weak formulation}
Let us consider the linearized problem \textcolor{blue}{(qua ci vuole la referenza a quella linearizzata)} in a more generalized form which reads as follows:
\begin{equation}
\left\{
\begin{array}{rcll}
\nabla \cdot (-\epsilon \nabla \varphi) + \sigma^{(k)}(\vect{x}) \varphi & = &  f^{(k)}(\vect{x}) & \psp{15} in \psp{2} \Omega \\
\varphi & = & \varphi_D & \psp{15} on \psp{2} \Gamma_D \\
\nabla \varphi \cdot \vect{n} & = & 0 & \psp{15} on \psp{2} \Gamma_N
\end{array}
\right.
\end{equation}

For the sake of simplicity we summerize the reaction and force term in $\sigma$ and $f$, but we kept visible the iteration dependence.
The well-posedness of such problem is ensured by several (and reasonable) hypotesis:
\begin{itemize}
\item $\epsilon \in L^{\infty}(\Omega)$ and $\exists m$ s.t. $0 < m \leq \epsilon$ (a.e.) in $\Omega$;
\item  $\sigma \in L^{\infty}(\Omega)$ and $\exists m$ s.t. $0 < m \leq \sigma$ (a.e.) in $\Omega$.
\end{itemize}

We proceed with the classical displacement weak formulation.
Given $\varphi_D \in H^{1/2}(\Gamma_D)$ and $f \in L^2(\Omega)$ find $\varphi \in H^1(\Omega)$ such that 

\begin{equation}
\int_{\Omega} \epsilon \nabla \varphi \nabla v \, d\Omega + \int_{\Omega} \sigma^{(k)}\varphi v \, d\Omega = \int_{\Omega} f^{(k)}v \, d\Omega \psp{15} \forall v \in H^1_{\Gamma_D}(\Omega)
\end{equation}

\subsection{Numerical approximation}

\subsection{Damping}
The main problem associated with the classical Newton method is the tendency to overestimate the length of the actual correction step for the iterate. This phenomenon is frequently termed overshoot. In the case of the semiconductor equations this overshoot problem has often been treated by simply limiting the size of the correction vector ($\delta \varphi$) determined by Newton's method. The usual established modifications to avoid overshoot are given by the seguent formulations:


\begin{align}
A(\varphi_k)&=\dfrac{1}{t_k}F'(\varphi_k) \label{eq: NLP mod used} \\
A(\varphi_k)&=s_kI+F'(\varphi_k) \label{eq: NLP mod not used}
\end{align}

$t_k$ and $s_k$ are properly chosen positive parameters. During the implementation of the code we chose \referenzaeq{eq: NLP mod used} method. Note that for $t_k=1$, $s_k=0$ these modified Newton methods reduce to the classical Newton method. We have now to deal with the question how to choose $t_k$ or $s_k$ that the modified Newton methods exhibit superior convergence properties compared to the classical Newton method.
For the case \referenzaeq{eq: NLP mod used} there's a simple criterion suggested by Deuflhard \textcolor{red}{referenza}: $t_k$ is taken from the interval $(0,1]$ in such a manner that for any norm,
\begin{equation}
\label{eq: extended criterion}
||F'(\varphi_k)^{-1}F(\varphi_k-t_kF'(\varphi_k)^{-1}F(\varphi_k))||<||F'(\varphi_k)F(\varphi_k)||
\end{equation}

Condition \referenzaeq{eq: extended criterion} guarantees that the correction of the k-th iterate is an improved approximation to the final solution, in other words the residual norm can only descents.
This condition can be easily evaluated only if the Jacobian matrix is factored into triangular matrices because the evaluation of the argument of the norm on the left hand side of \referenzaeq{eq: extended criterion} is then reduced to a forward and backward substitution and the evaluation of $F(\varphi)$. Although we use an iterative methods (BCG solver) which implies serious diffuclties to the application of the criterion. Another valid possibility is to use the main diagonal of $F'(\varphi_k)$, denoted as $D(\varphi_k)$:
\begin{equation}
\label{eq: easy criterion}
||D(\varphi_k)^{-1}F(\varphi_k-t_kD(\varphi_k)^{-1}F(\varphi_k))||<||F'(\varphi_k)F(\varphi_k)||
\end{equation}

This is the criterion developed in our code. However the value to use for $t_k$ is a question of trial and error. Frequently one chooses the following sequences:

\begin{align}
t_k & = \dfrac{1}{2^i} \\
t_k & = \dfrac{1}{2^{\dfrac{i(i+1)}{2}}}  
\end{align}

obiuvsly $i$ is the subiterations of damping reached when satisfied \referenzaeq{eq: easy criterion}. Sufficiently close to the solution \referenzaeq{eq: extended criterion} (and so \referenzaeq{eq: easy criterion}) will be satisfied with $t_k=1$ so that the convergence properties of the classical Newton method are recovered.
 